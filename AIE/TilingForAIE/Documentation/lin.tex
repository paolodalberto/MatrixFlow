

\documentclass[journal]{IEEEtran}
\IEEEoverridecommandlockouts
% The preceding line is only needed to identify funding in the first footnote. If that is unneeded, please comment it out.
%\usepackage{hyperref}
\usepackage{cite}
\usepackage{amsmath,amssymb,amsfonts}
\usepackage{algorithm}
\usepackage{algorithmic}
\usepackage{graphicx}
\usepackage{textcomp}
\usepackage{xcolor}
\usepackage{comment}
\def\BibTeX{{\rm B\kern-.05em{\sc i\kern-.025em b}\kern-.08em
    T\kern-.1667em\lower.7ex\hbox{E}\kern-.125emX}}
%\usepackage{}


\onecolumn 

\def\firmopsa{d}
\def\firmopsb{d}
\newcommand\AssignTags{TagIt}


% Definitions
% -----------
\def\x{{\mathbf x}}
\def\L{{\cal L}}
\def\spiral{SPIRAL\xspace}

\newcommand{\pred}{{}}

\newcommand{\tensor}[0]{\otimes}
\newcommand{\inv}[1]{{1{/}{#1}}}
%\newcommand{\inv}[1]{$\textstyle\frac{1}{#1}$}
\newcommand{\DFT}{\operatorname{\bf DFT}}
\newcommand{\WHT}{\operatorname{\bf WHT}}
\newcommand{\FIR}{\operatorname{\bf FIR}}
\newcommand{\one}[0]{{I}}
\newcommand{\eq}[0]{{=}}
\newcommand{\Tensor}[2]{#1{\tensor}#2}
\newcommand{\SPw}[2]{{\prec}#1,#2{\succ}}
\newcommand{\SPr}[2]{{<}#1,#2{>}}

\newcommand{\WTensorI}[2]{\Tensor{\WHT_{#1}}{\one_{#2}}}
\newcommand{\ITensorW}[2]{\Tensor{\one_{#1}}{\WHT_{#2}}}
\newcommand{\Wht}[2]{(\WTensorI{\WHT_{#1}}{#2})(\ITensorW{#1}{#2})}

\newcommand{\Times}[2]{{#1}{\times}{#2}}
\newcommand{\EQ}{{=}}
\newcommand{\Space}{{ S}}
\newcommand{\PEQ}{{{+}{=}}}
\newcommand{\SEQ}{{{-}{=}}}
\newcommand{\h}{\psi}
\newcommand{\balpha}{\bm{\alpha}}
\newcommand{\bgamma}{\bm{\gamma}}
\newcommand{\bbeta}{\bm{\beta}}
\newcommand{\bmu}{\bm{\mu}}
\newcommand{\bsigma}{\bm{\sigma}}


\newcommand{\f}{\varphi}
\newcommand{\Z}{\mathbb{Z}}
\newcommand{\A}{\mathbb{A}}
\newcommand{\B}{{\bf B}}
\newcommand{\X}{\mathbb{X}}
\newcommand{\T}{\mathbb{T}}
%\newcommand{\Q}{\mathbb{Q}}
\newcommand{\N}{\mathbb{N}}
\newcommand{\R}{\mathbb{R}}
\newcommand{\CC}{{\Gamma}}
\newcommand{\D}{\mathbb{D}}
\newcommand{\M}{\mathbb{M}}
\newcommand{\F}{\mathbb{F}}
\newcommand{\ST}{\mathbb{S}}
\newcommand{\Vc}[1]{{\boldsymbol #1}}
\newcommand{\fl}[1]{{\lfloor {#1} \rfloor}}
\newcommand{\cl}[1]{{\lceil {#1} \rceil}}
\newcommand{\half}[1]{{\frac{#1}{2}}}
\newcommand{\q}[1]{\Times{\cl{#1}}{\cl{#1}}}
\newcommand{\qq}[1]{\Times{\cl{#1}}{\fl{#1}}}
\newcommand{\qqq}[1]{\Times{\fl{#1}}{\cl{#1}}}
\newcommand{\qqqq}[1]{\Times{\fl{#1}}{\fl{#1}}}
\newcommand{\size}[1]{\sigma(#1)}
\newcommand{\Size}[3]{\sigma(\Vc{#1})\EQ\Times{#2}{#3}}
\newcommand{\n}{(\frac{n}{2})}
\newcommand{\m}{(\frac{n-1}{2})}
\newcommand{\Ceil}[1]{{\cl{\half{#1}}}}
\newcommand{\Floor}[1]{{\fl{\half{#1}}}}

\newcommand{\Endash}{{--}}
\newcommand{\Emdashbegin}{{---}}
\newcommand{\Emdashend}{{---}~}

\newcommand{\Transducer}[1] {{\mathbf #1 }}


\newcommand{\AO}  {\Times{\Ceil{m}}{\Ceil{n}}}
\newcommand{\AI} {\Times{\Ceil{m}}{\Floor{n}}}
\newcommand{\AII}{\Times{\Floor{m}}{\Ceil{n}}}
\newcommand{\AIII} {\Times{\Floor{m}}{\Floor{n}}}
\newcommand{\BO}  {\Times{\Ceil{n}}{\Ceil{p}}}
\newcommand{\BI} {\Times{\Ceil{n}}{\Floor{p}}}
\newcommand{\BII}{\Times{\Floor{n}}{\Ceil{p}}}
\newcommand{\BIII} {\Times{\Floor{n}}{\Floor{p}}}
\newcommand{\CO}  {\Times{\Ceil{m}}{\Ceil{p}}}
\newcommand{\CI} {\Times{\Ceil{m}}{\Floor{p}}}
\newcommand{\CII}{\Times{\Floor{m}}{\Ceil{p}}}
\newcommand{\CIII} {\Times{\Floor{m}}{\Floor{p}}}

\newcommand{\I}[2]{\Times{\Ceil{#1}}{\Ceil{#2}}}
\newcommand{\II}[2]{\Times{\Ceil{#1}}{\Floor{#2}}}
\newcommand{\III}[2]{\Times{\Floor{#1}}{\Ceil{#2}}}
\newcommand{\IV}[2]{\Times{\Floor{#1}}{\Floor{#2}}}


\newcommand{\Q}[2]{\Vc{#1}_{#2}}

\newcommand{\QAO}{{\Vc{A}_{0}}}
\newcommand{\QAI}{\Vc{A}_{1}}
\newcommand{\QAII}{\Vc{A}_{2}}
\newcommand{\QAIII} {\Vc{A}_{3}}

\newcommand{\QBO}  {\Vc{B}_{0}}
\newcommand{\QBI} {\Vc{B}_{1}}
\newcommand{\QBII}{\Vc{B}_{2}}
\newcommand{\QBIII} {\Vc{B}_{3}}

\newcommand{\QCO}  {\Vc{C}_{0}}
\newcommand{\QCI} {\Vc{C}_{1}}
\newcommand{\QCII}{\Vc{C}_{2}}
\newcommand{\QCIII} {\Vc{C}_{3}}
\newcommand{\FIGsize}[2]{{width={#1}, height={#2}}}

\newcommand{\Mod}[2]{{#1}{\text{ \bf mod }}{#2}}

\newcommand{\TableRef}[1]{Table \ref{#1}}
\newcommand{\FigureRef}[1]{Figure \ref{#1}}
\newcommand{\SectionRef}[1]{Section \ref{#1}}
\newcommand{\EquationRef}[1]{Equation \ref{#1}}
\newcommand{\OnPageRef}[1]{on page \pageref{#1}}

% Formatting
% ----------
%\setlength{\evensidemargin}{0mm}
%\setlength{\oddsidemargin}{0mm}
%\setlength{\textwidth}{6.5in}
%\setlength{\textheight}{9.5in}
%\setlength{\topmargin}{0in}
%\setlength{\headheight}{0in}
%\setlength{\headsep}{0mm}

\newenvironment{noinds_itemize}{\begin{list}{$\bullet$}
{\setlength{\rightmargin}{0em}
\setlength{\leftmargin}{1.2em}
\setlength{\itemsep}{0em}
\setlength{\topsep}{0em}
\setlength{\parsep}{0em}}}{\end{list}}



\newcommand{\mypar}[1]{{\bf #1.}}
\newcommand{\mytag}[2]{[#1]_{#2}}
\sloppy

%% jan 01, 2009 - dasdan - removed 'final' pdflatex cannot work with it.
%\usepackage[final=true,bookmarks=true,bookmarkstype=toc,colorlinks=true, linkcolor=blue,citecolor=blue]{hyperref}
%\usepackage[bookmarks=true,bookmarkstype=toc,colorlinks=true, linkcolor=blue,citecolor=blue]{hyperref}

% width fraction, mflops, relative, caption, label
\newcommand{\doublefigure}[5]{{\begin{figure}[htb]%
\centering %
\includegraphics[width=#1\linewidth]{#2}
\includegraphics[width=#1\linewidth]{#3}
\caption{#4}%bio                 
\label{#5}%
\end{figure}}}

\newcommand{\Doublefigure}[5]{{\begin{figure*} %
\centering %
\includegraphics[width=#1\linewidth]{#2}%
\includegraphics[width=#1\linewidth]{#3}%
\caption{#4}%
\label{#5}%
\end{figure*}}}

\newcommand{\Singlefigure}[4]{{\begin{figure*} %
\centering %
\includegraphics[width=#1\linewidth]{#2}%\figurebox{#1\linewidth}{}{}[#2]
\caption{#3}%
\label{#4}%
\end{figure*}}}

\newcommand{\singlefigure}[4]{{\begin{figure}[htb] %
\centering %
\includegraphics[width=#1\linewidth]{#2}%
%\includegraphics[height=#1\linewidth,angle=-90]{#3} %
%\figurebox{#1\linewidth}{}{}[#2]\\
\caption{#3}%
\label{#4}%
\end{figure}}}

\newcommand{\singlefigurerotate}[4]{{\begin{figure}[htb] %
\centering %
\includegraphics[width=#1\linewidth,angle=-90]{#2}%
%\includegraphics[height=#1\linewidth,angle=-90]{#3} %
%\figurebox{#1\linewidth}{}{}[#2]\\
\caption{#3}%
\label{#4}%
\end{figure}}}

\newcommand{\orthogonal}{%
\mathrel{\raisebox{.1em}{%     
\reflectbox{\rotatebox[origin=c]{90}{$\models$}}}}}


%\graphicspath{ {./}{./depth} }

\begin{document}

%%%%%%%%%%%---SETME-----%%%%%%%%%%%%%
\title{Tiling/Blocking MHA, GEMM, LN, CONV }

\author{
  \IEEEauthorblockN{P.D.}
}

%\renewcommand{\shortauthors}{D'Alberto, et al.}
\maketitle

\begin{abstract}
  How a conversation with Zhongexlive-latex-base, texlive-fonts-recommendedand texlive-latex-recommendedyi Lin about Multi-head attention
  became the starting point for a stand alone tiling tool for the
  search, description and computation of blocked algorithms for AIE
  overlays.

  We start with a formal description of MHA and we jot down several
  ideas how to describe, code, and interpret tiling strategies. This
  is our first attempt to describe formally and optimally the tiling
  process. We started with pure tiling as suggestions to an FPGA IP
  (XDNN), with layer fusion and explicit double buffering by software
  pipelining (DPUV3INT8). We introduce explicit code and compute
  parallelism by symmetric computations for sparse convolutions
  (Weight block sparsity). Here, we finally explore all valid
  solutions and optimum by any custom function with the ability to
  represent complex computation using double buffering at all level of
  the memory hierarchy without being hold back by how to implement
  each instructions (but having done so in the past).
  


  
\end{abstract}


\begin{IEEEkeywords}
 AI, FPGA, Performance, LLM BS and Tools
\end{IEEEkeywords}

\section{MHA, its Notations and an Introduction}
\label{sec:introduction}
This is a HEAD
\begin{equation}
  \Vc{R} = \Smax{\Vc{Q}*\Vc{K}^t}*\Vc{V}
\end{equation}

Where $\Vc{K},\Vc{Q},\Vc{V} \in \R ^{L\times d}$, where $d\in\{32,64\}$ is small and $L \in \{4096\}$
is large
\begin{equation}
  \Smax{\Vc{M}}  = \Vc{S}, \text{ with } \Vc{S}, \Vc{M} \in R^{m\times n} \\
\end{equation}

\begin{equation}
  \Smax{\Vc{M}}_{i,j}  = \Smaxd{\Vc{M}} 
\end{equation}
Where the denominator is a sum by row and the result is literally a
vector normalizing the row of hte numerator. Another way is to
consider it a unitary matrix multiplying on the left
\begin{equation}
  \label{eq:gamma}
  \Gamma= \frac{1}{\sum_j\exp{M_{i,j}}} = [ \frac{1}{\sum_j\exp{M_{0,j}}}, \dots,  \frac{1}{\sum_j\exp{M_{m,j}}} ] * \Vc{I}
\end{equation}
\begin{equation}
  \Smax{\Vc{M}}  = \Gamma \Vc{M} = \Vc{S}
\end{equation}

\begin{equation}
  \Vc{Q}*\Vc{K}^t = \Vc{T} \in \R^{L\times L}
\end{equation}

The submatrix of $\Vc{K}^t$ are actually transposed, so I do not need
to carry the transpose sign everywhere. Assume that $\Vc{K}_i, \Vc{Q}_j \in \R^{d\times d}$
{\small \begin{equation*}
  \label{eq:mat}
  \begin{pmatrix}
    \Vc{Q}_0  \\
    \Vc{Q}_1  \\
    \dots \\
    \Vc{Q}_n
  \end{pmatrix} * 
  \begin{pmatrix}
    \Vc{K}_0 & \Vc{K}_1  & \dots \Vc{K}_n \\
  \end{pmatrix}\\
  =
  \begin{pmatrix}
    \Vc{Q}_0\Vc{K}_0 & \Vc{Q}_0\Vc{K}_1  & \dots \Vc{Q}_0\Vc{K}_n \\
    \Vc{Q}_1\Vc{K}_0 & \Vc{Q}_1\Vc{K}_1  & \dots \Vc{Q}_1\Vc{K}_n \\
    \dots \\
    \Vc{Q}_n\Vc{K}_0 & \Vc{Q}_n\Vc{K}_1  & \dots \Vc{Q}_n\Vc{K}_n \\
  \end{pmatrix}\\
\end{equation*}
}

\begin{equation*}
  \Vc{R}_0 = \Smax{
    \begin{pmatrix}
      \Vc{Q}_0\Vc{K}_0 &  \Vc{Q}_0\Vc{K}_1  & \dots & \Vc{Q}_0\Vc{K}_n 
    \end{pmatrix}
  }
  \begin{pmatrix}
    \Vc{V}_0  \\
    \Vc{V}_1  \\
    \dots \\
    \Vc{V}_n
  \end{pmatrix} 
\end{equation*}

To be precise, 

\begin{equation}
  \label{eq:a}
  \Vc{R}_0 = \frac{1}{\sum_j\exp{\Vc{Q}_0\Vc{K}_j}}
  \begin{pmatrix}
    \exp{\Vc{Q}_0\Vc{K}_0} &  \dots &\exp{\Vc{Q}_0\Vc{K}_n} 
  \end{pmatrix}
  \begin{pmatrix}
    \Vc{V}_0  \\
    \Vc{V}_1  \\
    \dots \\
    \Vc{V}_n
  \end{pmatrix} 
\end{equation}

\begin{equation}
  \label{eq:b}
  \Vc{R}_0 = \frac{\sum_j(\exp{\Vc{Q}_0\Vc{K}_j})\Vc{V}_j}{\sum_j\exp{\Vc{Q}_0\Vc{K}_j}}=\Gamma_0\sum_j(\exp{\Vc{Q}_0\Vc{K}_j})\Vc{V}_j
\end{equation}
The sum at the denominator is by row, the result is a vector and it
can be represented also as a left diagonal matrix multiplication using
Equation \ref{eq:gamma}.  The sum at the numerator is a matrix
addition (or a left matrix multiplication and thanks to
associative). Thus, Equations \ref{eq:a} and \ref{eq:b} are proper and
equivalent. We can apply to the remaining $\Vc{Q}_i$ and $\Vc{R}_i$.


\begin{comment}
\begin{equation*}
  \Vc{r}_{0,t-1} =
  \frac{\sum_{j=0}^{t-1}(\exp{\Vc{Q}_0\Vc{K}_j})\Vc{V}_j}{\sum_{j=0}^{t-1}\exp{\Vc{Q}_0\Vc{K}_j}}
\end{equation*}
\begin{equation*}
  \Vc{r}_{t,n} = \frac{\sum_{j=t}^{n}(\exp{\Vc{Q}_0\Vc{K}_j})\Vc{V}_j}{\sum_{j=t}^n\exp{\Vc{Q}_0\Vc{K}_j}}
\end{equation*}

\begin{equation}
  \Vc{D}_{\iota,\ell} = \sum_{j=\iota}^{\ell}\exp{\Vc{Q}_0\Vc{K}_j}
\end{equation}
\begin{equation}
  \Vc{D}_{0,n} = \Vc{D}_{0,t-1} + \Vc{D}_{t,n}
\end{equation}

\begin{equation*}
  \Vc{R}_0 = \frac{\Vc{D}_{0,t-1}}{\Vc{D}_{0,t-1} + \Vc{D}_{t,n}}\Vc{r}_{0,t-1}  +  \frac{\Vc{D}_{t,n}}{\Vc{D}_{0,t-1} + \Vc{D}_{t,n}}\Vc{r}_{t,n} 
\end{equation*}
The fraction is actually a scalar operation: it is an element wise
scalar to be multiplied to $\Vc{r}_{0,i}$

\begin{equation*}
  \Vc{r}_{0,t+1} = \frac{\Vc{D}_{0,t-1}}{\Vc{D}_{0,t-1} + \Vc{D}_{t,t+1}}\Vc{r}_{0,t-1}  +  \frac{\Vc{D}_{t,t+1}}{\Vc{D}_{0,t-1} + \Vc{D}_{t,t+1}}\Vc{r}_{t,t+1} 
\end{equation*}

The Division does not need to be done step by step but only at the
end. So we have a matrix update iteration as in Equation \ref{eq:sum0}
and \ref{eq:sum1}, and we can conclude with the final result in
Equation \ref{eq:res} and \ref{eq:resall}.
\end{comment}


\section{MHA, Blocked, Iterative Method by columns, and caching}
\label{sec:iterative}
Consider only the contribution by $\Vc{Q}_m$ and take $\Vc{N}_{m,0}=
\Vc{D}_{m,0}=0$, $\Vc{M}_{m,0} = \Vc{0}$, assume we have done the
computation until step $t-1$ and compute $t$ to $t<n$ then

\begin{align*}
  \Vc{M}^1  &= \max_{row}(\Vc{Q}_m\Vc{K}_{t},\Vc{M}_{m,t-1}) \\
  \Vc{E}    & =\exp{(\Vc{Q}_m\Vc{K}_{t}-\Vc{M}^1)} \\
  \Vc{S}    & =\exp{(\Vc{M}^1 - \Vc{M}_{m,t-1})} \\
  \Vc{M}_{m,t}   & =\Vc{M}^1 \\
\end{align*}
\begin{align}
  \label{eq:sum0}
  \Vc{N}_{m,t}  &=( \frac{\Vc{N}_{m,t-1}}{\Vc{S}}  + \Vc{E}\Vc{V}_{t}\  ) \\
  \label{eq:sum1}
  \Vc{D}_{m,t} &= \frac{\Vc{D}_{m,t-1}}{\Vc{S} +\Vc{E}} 
\end{align}

Finally, for $t=n$ and for every element of $\Vc{Q}_m$ we can repeat
the process and compute the final result
\begin{equation}
  \label{eq:resall}
 \forall i \in [0,n],\/ \Vc{R}_i = \frac{\Vc{N}_{i,n}}{\Vc{D}_{i,n}}
\end{equation}
\begin{comment}
Each $\Vc{R}_i$ are independent.  Space requirements: $\Vc{Q}_i,
\Vc{K}_i, \Vc{V}_i \in \R^{d \times d}$,, $\Vc{Q}_i\Vc{K}_j \in
\R^{d\times d}$, $\exp\Vc{Q}_i\Vc{K}_j \in \R^{d\times d}$
$\Vc{D}_{0,i} \in \R^{d}$, $\Vc{N}_{0,t}\Vc{N}_{0,t+1} \in \R^{d
  \times d}$.

We need to keep in memory $\Vc{D}_{0,t}$ and $\Vc{N}_{0,t}$, we
compute $\exp{\Vc{Q}_0\Vc{K}_{t+1}}$ to add to $\Vc{D}_{0,t}$, we then
compute $(\exp{\Vc{Q}_0\Vc{K}_{t+1}})\Vc{V}_{t+1}$. We need 5 matrices
of size $d \times d$. We store $\Vc{D}_{0,t+1}$ and $\Vc{N}_{0,t+1}$.
\end{comment}

\subsection{Caching}
Assume, we have $\Vc{K},\Vc{Q},\Vc{V} \in \R ^{L\times d}$, we split
them into partitions and we have $n$ of them.  We compute the result
$\Vc{R}$ as above Equation \ref{eq:resall} and we remember the result
$\Vc{R}$, the max vector $\Vc{M} = \{\Vc{M}_i\}$, the scaling diagonal
$ \Vc{D} = \{ \Vc{D}_{i} \}$, and the numerators
$\{\Vc{N}_{i}\}$. The last two we remember just the last {\em column
  computation}.

Now, consider the case that we read
$\Vc{K}_{n+1},\Vc{Q}_{n+1},\Vc{V}_{n+1}$ and we want to compute the
update version $\Vc{R}$. We have two observa!tions: first, we need to
compute a fresh $\Vc{R}_{n+1}$ from scratch:
\begin{align}
  \Vc{R}_{n+1},\Vc{M}_{n+1},\Vc{N}_{n+1}, \Vc{D}_{n+1} &= \Smax{\Vc{Q}_{n+1}*\Vc{K}^t}*\Vc{V}.
\end{align}
Second, the update of $\Vc{R}_i$ with $i<n+1$ is exactly the same
Equation \ref{eq:sum0}, \ref{eq:sum1}, and \ref{eq:resall}.



\subsection{Range of $\exp(\Vc{Q}_0\Vc{K}_j)$}
\label{sec:range}
The function $e^x$ has the property that $\frac{\partial e^x}{\partial
  x}= e^x$. The function $e^x$ is always positive, is always
increasing, and $e^0=1$ and also the derivative. For $x>0$ and with
numerical representation that could be limited in range, we want to
stay on the left side of $x=0$ or at a very least in a bound region to
avoid saturation and overflow. Clearly, for $x>0$ we have
$\frac{\partial e^x}{\partial x}>1$ and it really highlights small
variation of $x$.

If we keep the range of $\Vc{Q}_0\Vc{K}_j$ being smaller than one,
then the $\exp()$ is bound to $e$. Due to the nature of the
exponential function, the range will be positive and increasing (well
exponentially). For example, assume we compute the product
$\Vc{Q}\Vc{K}$ and we determine the maximum by row. This is a column
vector $\Vc{M} = \max_{row} \Vc{Q}\Vc{K}$ then we can compute the
final result. 

\begin{equation}
  \label{eq:c}
  \Vc{R}_0^t = \frac{\sum_j(\exp{(\Vc{Q}_0\Vc{K}_j-\Vc{M})})\Vc{V}_j}{\sum_j\exp{(\Vc{Q}_0\Vc{K}_j-\Vc{M})}} =
     \Vc{R}_0\frac{e^{\Vc{M}}}{e^{\Vc{M}}}= \Vc{R}_0
\end{equation}
The exponential range is under control, the summation at the
denominator is increasing but bound to the number of additions.  The
denominator can be applied at the end of the computation but the
normalization of the range as to be applied as soon as possible in
such a way to have undesired overflow. In Section \ref{sec:iterative}, we combine
everything into an iterative method that can be split into independent
computations.

\subsection{The identity element in $\Smax{\Vc{M}}$}
\label{sec:identity}

In the ring based on the operation addition $+$, there is a identity
element $0$ (zero) such as $\forall a, a+0 = 0+a = a$. For
multiplication $*$ the identity element is $1$ (one). If the operands
are matrices, we can extend the zero and one elements to zero matrices
and to diagonal one matrices.

Take $\Smaxd{\Vc{M}}$, if for any reason we need to pad or extend
$\Vc{M}$ to $\Vc{N}$, what can we do so that $\Smaxd{\Vc{M}}=
\Smax{\Vc{N}}$? How do we pad it?

\begin{itemize}
  \item {\bf Padding the inputs before matrix multiplication will not
    work.} For example, take the last $\Vc{K}_n \in d\times m$ and for
    the computation of the matrix multiplication $\Vc{Q}_0\Vc{K}_n$ we
    must increase $m$ to $m+1$ (say, alignment properties). This may
    affect also $\Vc{V}_n$.  This will change the shape of the results
    and in practice, we would introduce $d$ zeros into the matrix
    results. This will introduce 1 into the matrix
    $\exp(\Vc{Q}_0\Vc{K}_n)$ and when applied to
    $\Smax{\Vc{Q}_0\Vc{K}_n}$, then we change the shape of the
    numerator and we increase the denominator (by one in this
    case). If we need to compute the max for range purpose as in
    Equation \ref{eq:c}, we are changing also the values of the
    numerator matrix and given the $\max$ is not bijective, we cannot
    reverse the effects. Padding the operands and in combination with
    range constraints, we cannot have a clean computation.

  \item {\bf Padding the output of $\exp(\Vc{M})$ with zeros.}  In
    practice, the unitary element of the denominator sum is zero
    $\exp(-\infty)$, the unitary element for $\max$ is $-\infty$.
    Either pad the output with zero or the input with $-\infty$. In
    finite precision and for $\max$, any $\min$ instead of $-\infty$
    value will work, however, the $\exp(\min)$ will not translate to
    zero in floating point representation unless we fix the result to
    zero.

    How do we introduce $-\infty$ ? or how we introduce zeros to a
    specific output pattern?

 
    
    

\end{itemize}
















\subsection{Implementation of the head}

We discuss here the case where where we perform the operations with
{\em int16}. Consider $L=512$, we split the matrices $\Vc{Q}$ and
$\Vc{V}$ into horizontal parts and we split $\Vc{K}^t$ into vertical
parts. We choose to split $\Vc{K}^t$ and $\Vc{V}$ equally into four
core so each core will receive $2^6\times
\lceil\frac{2^9}{2^2}\rceil=2^7$ for a space of $2^{14}$ Bytes and we
need to store the same amount of $\Vc{V}$ for a total space of
$2^{15}$ this is equivalent to four banks. We can split the
computation so that $\Vc{V}_i, \Vc{K}_i \in 64\times 64$ and thus we
send into each core 2 $\Vc{V}_i$ and 2 $\Vc{K}_i$.

Let consider what should be the size of $\Vc{Q}_0 = m \times 64$ so
that we can fit a basic computation in a core.

$\Vc{T} = \Vc{Q}_0\Vc{K}_i = m \times 64$, 
$\Vc{T} = \exp{\Vc{T}} = m \times 64$,
$\Vc{D} += \Vc{T}$, $\Vc{T} = \Vc{T}\Vc{V}_i$, and $\Vc{N} +=
\Vc{T}$. We need to have space for 4 matrices matrices of size $ m
\times 64$. Thus $m2^32^6$ must fit into 2 banks $2^{14}$.  We have $m
= 2^5 = 32$.

Core $A_i$ compute $N_{i*2,(i+1)*2 -1}$ and $D_{i*2,(i+1)*2 -1}$ With
three reductions per column we have the computation of $R_0$, we
repeat the process.  The subvolume for $\Vc{Q}_i = 32\times 64$.  We
need to broad cast $\Vc{Q}_i$ 16 times.

\singlefigure{0.20}{20240222_214743.jpg}{HEAD for int16}{fig:head}


In this section, we show the algorithm that can be implemented on a
$2\times 4$ having the problem size $d=64$ and $L=512$


\singlefigure{0.40}{20240217-151903.jpg}{HEAD for int 8}{fig:head1}

\subsection{The curious case of the problem size $l=77$ and $L=768$}

This is the place holder for the curious case and may be a code
generator. 


\newpage
\section{GEMM for Asymmetric systolic arrays}

We start a conversation about $\Vc{C} = \Vc{A}*\Vc{B}$ where we imply
that the right operand is larger than the left operand. If it may
happen the opposite, then we may want to compute $\Vc{C}^t =
\Vc{B}^t*\Vc{A}^t$, the shapes may be different but the sizes stay the
same. The algorithm in a nut shell is a systolic algorithm where the
matrix C is spatially separated and computed completely before
written. Any matrices we are going to present in this section will
have the usual notation

\begin{equation*} \Vc{M} =
    \begin{pmatrix}
    \Vc{M}_{0,0} & \Vc{M}_{0,1} & \Vc{M}_{0,2} & \Vc{M}_{0,3} & \dots \\
    \Vc{M}_{1,0} & \Vc{M}_{1,1} & \Vc{M}_{1,2} & \Vc{M}_{1,3} & \dots \\
    \Vc{M}_{2,0} & \Vc{M}_{2,1} & \Vc{M}_{2,2} & \Vc{M}_{2,3} & \dots \\
    \Vc{M}_{3,0} & \Vc{M}_{3,1} & \Vc{M}_{3,2} & \Vc{M}_{3,3} & \dots \\
    \dots
  \end{pmatrix}  
\end{equation*}
Where each element is a sub matrix. We will refer these as subvolumes
but they can be reduce to scalar if necessary.  We assume an array of
AIEs in a formation $4 \times 2$. Two columns and four rows.

{\small \begin{equation}
  \label{eq:gemm-fatc}
  \begin{pmatrix}
    \Vc{C}_{0,0} = \sum_k \Vc{A}_{0,k}\Vc{B}_{k,0} & \Vc{C}_{1,0} \sum_k \Vc{A}_{1,k}\Vc{B}_{k,0}\\
    \Vc{C}_{0,1} = \sum_k \Vc{A}_{0,k}\Vc{B}_{k,1} & \Vc{C}_{1,1} \sum_k \Vc{A}_{1,k}\Vc{B}_{k,1}\\
    \Vc{C}_{0,2} = \sum_k \Vc{A}_{0,k}\Vc{B}_{k,2} & \Vc{C}_{1,2} \sum_k \Vc{A}_{1,k}\Vc{B}_{k,2}\\
    \Vc{C}_{0,3} = \sum_k \Vc{A}_{0,k}\Vc{B}_{k,3} & \Vc{C}_{1,3} \sum_k \Vc{A}_{1,k}\Vc{B}_{k,3}\\
  \end{pmatrix}  
\end{equation}
}

$\Vc{A}_{0,k}$ is broad cast on column zero and $\Vc{A}_{1,k}$ to
column one, one subvolume at a time. $\Vc{B}_{k,0}$ is broad cast on
row zero and $\Vc{B}_{k,3}$ on row three, one subvolume at a
time. This is a systolic communication.

\subsection{Sub-Volume}
The basic computation for one core is $\Vc{C}_{i,j} = \sum_k
\Vc{A}_{i,k}\Vc{B}_{k,j}$, the basic subvolume for this computation is
$\Vc{C}_{i,j}, \Vc{A}_{i,k},\Vc{B}_{k,j}$. They are not same-size matrices. They need to be in the AIE cores. This has eight banks of 8
KB each ($2^{13}$ elements).  We assume that the computation and
communication to the core is double buffered.
\begin{algorithm}
  \caption{Core streaming computation for $\Vc{C}_{0,0}$}
  \label{alg:one}
  \begin{algorithmic}[1]
    \STATE $\Vc{T}_A = \Vc{A}_{0,0}, \Vc{T}_B = \Vc{B}_{0,0}$ 
    \STATE $\forall i [1,K-1], \Vc{{TT}}_A = \Vc{A}_{0,i} , \Vc{{TT}}_B = \Vc{B}_{i,0},    \Vc{C}_{0,0} += \Vc{T}_{A}\Vc{T}_{B}, \Vc{T}_A = \Vc{{TT}}_{A} , \Vc{T}_B = \Vc{{TT}}_B$ 
    \STATE $\Vc{C}_{0,0} += \Vc{T}_{A}\Vc{T}_{B}$ 
  \end{algorithmic}  
\end{algorithm}

We decide sub-volume $\Vc{C}_{0,0}$ will fit a bank $2^{13}$, to keep
the most balance computation we choose $(128=2^7) \times (64=2^6)$ (8
KB). This means that $\Vc{A}_{0,0}$ will be $128 \times 64$ (8 KB) and
$\Vc{B}_{0,0}$ will be $64 \times 64$ (4 KB).

One step (tick) of the overlay computation is an {\em outer} product

{\small \begin{equation}
  \label{eq:gemmA}
  \begin{pmatrix}
    \Vc{C}_{0,0}  &  \Vc{C}_{0,1}  & \Vc{C}_{0,2}  &\Vc{C}_{0,3} \\
    \Vc{C}_{1,0}  & \Vc{C}_{1,1}   & \Vc{C}_{1,2}  & \Vc{C}_{1,3} \\
  \end{pmatrix}    =
  \begin{pmatrix}
    \Vc{A}_{0,k}  \\
    \Vc{A}_{1,k}   \\
  \end{pmatrix}  
  \begin{pmatrix}
    \Vc{B}_{k,0}  &  \Vc{B}_{k,1}  & \Vc{B}_{k,2}  &\Vc{B}_{k,3} \\
  \end{pmatrix}    
\end{equation}
} We read 16 KB of $\Vc{A}$ ($256 \times 64$) , we read 16 KB of
$\Vc{B}$ ($64 \times 256$), and we compute $256 \times 256$ elements
of $\Vc{C}$. In practice, the full computation is formally, where we
introduce a step in the outer loop:




{\small \begin{equation*}
    %\label{eq:gemm}
    \sum_{i=0,2}^M
    \sum_{j=0,4}^N
  \begin{pmatrix}
    \Vc{C}_{i,j}  & \Vc{C}_{i,j+1}    & \Vc{C}_{i,j+2}    &\Vc{C}_{i,j+3} \\
    \Vc{C}_{i+1,j}  & \Vc{C}_{i+1,j+1}   & \Vc{C}_{i+1,j+2}  & \Vc{C}_{i+1,j+3} \\
  \end{pmatrix}    = \sum_k^K
  \begin{pmatrix}
    \Vc{A}_{i,k}  \\
    \Vc{A}_{i+1,k}   \\
  \end{pmatrix}  
  \begin{pmatrix}
    \Vc{B}_{k,j}  &  \Vc{B}_{k,j+1}  & \Vc{B}_{k,j+2}  &\Vc{B}_{k,j+3} \\
  \end{pmatrix}    
\end{equation*}
}



We have a ratio compute over communication per core
\begin{equation}
  \label{eq:ratio}
  \F = \frac{2^7 * 2^6 * 2^7 } { 2* 2^7 * 2^6} = \frac{2^{20}}{2^{14}} = 2^6
\end{equation}
This ratio is useful to estimate the smallest subvolume we can afford
in order to hide communication latency by double buffering or figure
out the ratio between the computation speed and bandwidth. 

For example, assume we can send each operand by different channels at
8 GB/s = $2^3 *2^{30}$ and 256 operations can be done in one cycle at
1 GHz. The ratio above will become

\[
\F_c = \frac{2^7 * 2^6 * 2^7 * 2^{33}} { 2* 2^7 * 2^6* 2^8 *10^9} \sim \frac{2^{53}}{2^{14+8+28=50}} > 2^3  
\]
For any values larger than one, the execution time is compute
bound. In the same way we can choose the operand sizes:

\[
\frac{2^3 * 2^3 * 2^3 * 2^{33}} { 2* 2^3 * 2^3* 2^8 *10^9} \sim \frac{2^{42}}{2^{7+8+28=43}} >1/2    
\]
any sub volume smaller than $8 \times 8$ will be communication bound
and double buffering will not help. This Ratio can be used for the
investigation of double buffering and latency of reading operands in
Mem-Tiles.

\subsubsection{Transpose Algorithm}
Because the sizes and shapes of the operands does not need to be the
same, we may need to consider also the transpose algorithm

{\small \begin{equation}
  \label{eq:gemmB}
  \begin{pmatrix}
    \Vc{C}_{0,0}  & \Vc{C}_{0,1}  \\
    \Vc{C}_{1,0}  & \Vc{C}_{1,1}    \\
    \Vc{C}_{2,0}  & \Vc{C}_{2,1}  \\
    \Vc{C}_{3,0}  & \Vc{C}_{3,1} \\
  \end{pmatrix}    =
  \begin{pmatrix}
    \Vc{A}_{0,k}  \\
    \Vc{A}_{1,k}  \\
    \Vc{A}_{2,k}  \\
    \Vc{A}_{3,k}  \\
  \end{pmatrix}  
  \begin{pmatrix}
    \Vc{B}_{k,0}  &  \Vc{B}_{k,1}  \\
  \end{pmatrix}    
\end{equation}
} We read 16 KB of $\Vc{A}$ ($64 \times 256$) , we read 16 KB of
$\Vc{B}$ ($256 \times 64$), and we compute $256 \times 256$ elements
of $\Vc{C}$. In practice, the full computation is formally, where we
introduce a step in the outer loop: 

{\small \begin{equation*}
    %\label{eq:gemm}
    \sum_{i=0,4}^M
    \sum_{j=0,2}^N
  \begin{pmatrix}
    \Vc{C}_{i,j}  & \Vc{C}_{i,  j+1}  \\
    \Vc{C}_{i+1,j} & \Vc{C}_{i+1,j+1} \\
    \Vc{C}_{i+2,j} & \Vc{C}_{i+2,j+1} \\
    \Vc{C}_{i+3,j} & \Vc{C}_{i+3,j+1} \\ 
  \end{pmatrix}    = \sum_k^K
  \begin{pmatrix}
    \Vc{A}_{i,  k}  \\
    \Vc{A}_{i+1,k}   \\
    \Vc{A}_{i+2,k}  \\
    \Vc{A}_{i+3,k}   \\
  \end{pmatrix}  
  \begin{pmatrix}
    \Vc{B}_{k,j}  &  \Vc{B}_{k,j+1} \\
  \end{pmatrix}    
\end{equation*}
}

The main difference and utility to have two algorithms (for
memtile-core) is the better way to have a different ratio of the
computation. As a practical point, Equation \ref{eq:gemmA} leads to a
reuse of $\Vc{A}$ in memtile (if $\Vc{A}$ smaller space) and Equation
\ref{eq:gemmB} leads to a reuse of $\Vc{B}$ (if $\Vc{B}$ is smaller).
This will be evident when we write and present a code generator based
on this document. For


\subsubsection{Different sizes: int16}
In the case, we are working with an element 2 Bytes (int 16) and we
can keep the same subvolumes, the fraction $\F = \frac{2^6}{2}$. The
number of computation does not change, the communication increases
linearly.

The sub-volume computations has to change. We take $Vc{C}_{0,0} =
64\times 64 = 2^6 \times 2^6$, this means of size $2*2^12$, this is 8
KB (again). $\Vc{A}_{0,k}, \Vc{B}_{k,0} = 64 \times 64$, one bank
again. The subvolume computation is $64\times 64 \times 64$. Changing
the subvolume does not affect the Bytes transfer per tick (16 KB for
each operand). This is still balanced.



\subsection{Mem-tiles and the case   $A=512\times 768$ $B=768\times 768$ }

The matrix $\Vc{A}$ has size $(512=2^9) \times (768= 3*2^8)$ that is
$3*2^{17}$ and this fits nicely in a memtile of size 512 KB, that is
$2^{19}$ elements.  The matrix $\Vc{B}$ has size $768 \times 768$ that
is $9*2^{16}>2^{19}$. The nice property is that we can split the
matrix $\Vc{B}$ by column into three parts exactly and we can do
double buffering


{\small \begin{equation*}
    %\label{eq:gemm}
    \sum_{j=0,4}^{12}
    { \{ } \sum_{i=0,2}^4
  \begin{pmatrix}
    \Vc{C}_{i,j}  & \Vc{C}_{i,j+1}    & \Vc{C}_{i,j+2}    &\Vc{C}_{i,j+3} \\
    \Vc{C}_{i,j}  & \Vc{C}_{i+1,j+1}   & \Vc{C}_{i+1,j+2}  & \Vc{C}_{i+1,j+3} \\
  \end{pmatrix}    = \sum_k^{12}
  \begin{pmatrix}
    \Vc{A}_{i,k}  \\
    \Vc{A}_{i+1,k}   \\
  \end{pmatrix}  
  \begin{pmatrix}
    \Vc{B}_{k,j}  &  \Vc{B}_{k,j+1}  & \Vc{B}_{k,j+2}  &\Vc{B}_{k,j+3} \\
  \end{pmatrix}
  { \}}    
\end{equation*}
}

The curly brackets show the part of the computation when the sub
matrix of $\Vc{B}$ of size $768 \times 256$ is double buffered.

\subsection{Mem-tiles and the case  int16  $A=512\times 768$ $B=768\times 768$ }

The matrix $\Vc{A}$ has size $(512=2^9) \times (768= 3*2^8)$ that is
$2*3*2^{17}$, We need to tile and double buffering $\Vc{A}$.  We need
need to feed two columns: $A_{0,*}$ and $A_{1,*}$ at a time. One ping
per column will be $2*(64 \time 768)$: If we split $\Vc{A}$ into 4
tiles of 128 rows (4 subvolumes), We can have 4 iterations where we
double buffer $A_{i,*}$ and $A_{i,*}$. We need to keep a space of 384
KB in memtile 0.



The matrix $\Vc{B}$ has size $768 \times 768$ that is
$2*9*2^{16}>2^{19}$. If we keep the same sub-volume, we need
$\Vc{B}_{*,0} \dots \Vc{B}_{*,3}$ thus $2*256 \times 768$, four double
buffering we need twice as much as that. To make this possible we need
to change the subvolume of $\Vc{B}$ and of $\Vc{C}$.

To allow double buffering of the $\Vc{B}$ in memtile, The sub-volume
computations has to change. We take $Vc{C}_{0,0} = 64\times 32 = 2^6
\times 2^5$, this means of size $2*2^{11}$, this is 4 KB
(again). $\Vc{A}_{0,k} = 64 \times 64$ one bank, $\Vc{B}_{k,0} = 64
\times 32$, 4 KB. The subvolume computation is $64\times 64 \times
32$. Changing the subvolume does affect the Bytes transfer per tick
(16 KB for A and 8KB for B). 




{\small \begin{equation*}
    %\label{eq:gemm}
    \sum_{j=0,4}^{24}
    { \{ } \sum_{i=0,2}^8 
  \begin{pmatrix}
    \Vc{C}_{i,j}  & \Vc{C}_{i,j+1}    & \Vc{C}_{i,j+2}    &\Vc{C}_{i,j+3} \\
    \Vc{C}_{i,j}  & \Vc{C}_{i+1,j+1}   & \Vc{C}_{i+1,j+2}  & \Vc{C}_{i+1,j+3} \\
  \end{pmatrix}    = \sum_k^{12}
  \{
  \begin{pmatrix}
    \Vc{A}_{i,k}  \\
    \Vc{A}_{i+1,k}   \\
  \end{pmatrix}  
  \begin{pmatrix}
    \Vc{B}_{k,j}  &  \Vc{B}_{k,j+1}  & \Vc{B}_{k,j+2}  &\Vc{B}_{k,j+3} \\
  \end{pmatrix}
  \}{ \}}    
\end{equation*}
}
\begin{algorithm}
  \caption{Small Small still need tiling}
  \label{alg:one3}
  \begin{algorithmic}[1]
    \FOR{i=0 \TO 8 step 2}
      \STATE{Ping/Pong  $\Vc{A}_{i,*} .. \Vc{A}_{i+1,*}$}
      \STATE{\# Reuse A but hide latency}
      \FOR{j=0 \TO 24 step 4}
         \STATE{Ping/Pong  $\Vc{B}_{*,j} \dots \Vc{B}_{*,j+3}$}
         \STATE{$
           \begin{pmatrix}
    \Vc{C}_{i,j}  & \Vc{C}_{i,j+1}    & \Vc{C}_{i,j+2}    &\Vc{C}_{i,j+3} \\
    \Vc{C}_{i,j}  & \Vc{C}_{i+1,j+1}   & \Vc{C}_{i+1,j+2}  & \Vc{C}_{i+1,j+3} \\
  \end{pmatrix} = \sum_k^{12}
           \begin{pmatrix}
             \Vc{A}_{i,k}  \\
             \Vc{A}_{i+1,k}   \\
           \end{pmatrix}  
           \begin{pmatrix}
             \Vc{B}_{k,j}  &  \Vc{B}_{k,j+1}  & \Vc{B}_{k,j+2}  &\Vc{B}_{k,j+3} \\
         \end{pmatrix}$} 
      \ENDFOR
    \ENDFOR
  \end{algorithmic}  
\end{algorithm}

\[
\F_{cm} = \frac{2^6 * 2^6 * 2^5/128/10^9 } {2(2^{12}+2^{11})/8/2^{30}}=  0.75  
\]

This is communication bound. If we consider the computation of a core
and the time to read a ping of $\Vc{B}$ from DDR to memtile
(considering we can use 2 channels 8 GBs). We can reach break even
with the computation if we use 4 channels (copy to both
Memtiles)
\[
\F_{md} = \frac{2^6 * 3*2^8 * 2^5 /128/10^9  } {2*2^7*32^8/(2*8*2^{30}} = 0.53
\]

If we assume we can use 4 channels for DDR. The execution time will be
$6*8*12*{2(2^{12}+2^{11})/8/2^{30}}$ which is 411 us




\subsection{Mem-tiles and the case   $A=512\times 768$ $B=768\times 768*4$ }
The matrix $\Vc{A}$ has size $(512=2^9) \times (768= 3*2^8)$ that is
$3*2^{17}$ and this fits nicely in a memtile of size 512 KB, that is
$2^{19}$ elements.  The matrix $\Vc{B}$ has size $768 \times 768*4$
that is $9*2^{18}>2^{19}$. The nice property is that we can split the
matrix $\Vc{B}$ by column into 12 parts exactly and we can do double
buffering 

{\small \begin{equation*}
    %\label{eq:gemm}
    \sum_{j=0,4}^{48}
    { \{} \sum_{i=0,2}^4
  \begin{pmatrix}
    \Vc{C}_{i,j}  & \Vc{C}_{i,j+1}    & \Vc{C}_{i,j+2}    &\Vc{C}_{i,j+3} \\
    \Vc{C}_{i,j}  & \Vc{C}_{i+1,j+1}   & \Vc{C}_{i+1,j+2}  & \Vc{C}_{i+1,j+3} \\
  \end{pmatrix}    = \sum_k^{12}
  \begin{pmatrix}
    \Vc{A}_{i,k}  \\
    \Vc{A}_{i+1,k}   \\
  \end{pmatrix}  
  \begin{pmatrix}
    \Vc{B}_{k,j}  &  \Vc{B}_{k,j+1}  & \Vc{B}_{k,j+2}  &\Vc{B}_{k,j+3} \\
  \end{pmatrix}
  { \}}    
\end{equation*}
}


\subsection{Mem-tiles and the case $A=512\times 768*4$ $B=768*4 \times 768$ }
The matrix $\Vc{A}$ has size $(512=2^9) \times (768= 3*2^{10})$ that
is $3*2^{19}$.  The matrix $\Vc{B}$ has size $768*4 \times 768$ that
is $9*2^{18}>2^{19}$.

We present two algorithms. The first algorithm change the subvolume so
that we can double buffer A and reuse a slice of A multiple times.

Consider first that a matrix of size $512 \times 768$ fits a Memtile.
A slice of $\Vc{A}$ of size $64 \times 768*4$ allows a double
buffering. This means that a column will solve a subvolume of size
$m=32$ (64 divided by 2). Equivalently, a slice of $\Vc{B}$ of size
$t68*4 \times 64$ will allow double buffering and thus a sub volume of
$n=16$ (64 divided by 4). 

$\Vc{C}_{i,j}$ has size $32 \times 16$.   $\Vc{A}_{i,k}$ has size $32 \times 64$. $\Vc{B}_{k,j}$ has size $64 \times 16$.      

{\small \begin{equation*}
    %\label{eq:gemm}
    \sum_{i=0,2}^{8}
    { \{} \sum_{j=0,4}^{96}
  \begin{pmatrix}
    \Vc{C}_{i,j}  & \Vc{C}_{i,j+1}    & \Vc{C}_{i,j+2}    &\Vc{C}_{i,j+3} \\
    \Vc{C}_{i,j}  & \Vc{C}_{i+1,j+1}   & \Vc{C}_{i+1,j+2}  & \Vc{C}_{i+1,j+3} \\
  \end{pmatrix}    = \sum_k^{192}
  \{
  \begin{pmatrix}
    \Vc{A}_{i,k}  \\
    \Vc{A}_{i+1,k}   \\
  \end{pmatrix}  
  \begin{pmatrix}
    \Vc{B}_{k,j}  &  \Vc{B}_{k,j+1}  & \Vc{B}_{k,j+2}  &\Vc{B}_{k,j+3} \\
  \end{pmatrix}
  \}
  { \}}    
\end{equation*}
}

\begin{algorithm}
  \caption{Large Large}
  \label{alg:one2}
  \begin{algorithmic}[1]
    \FOR{i=0 \TO 8 step 2}
      \STATE{Ping/Pong  $\Vc{A}_{i,*} .. \Vc{A}_{i+1,*}$}
      \STATE{\# Reuse A but hide latency}
      \FOR{j=0 \TO 48 step 4}
         \STATE{Ping/Pong  $\Vc{B}_{*,j} \dots \Vc{B}_{*,j+3}$}
         \STATE{$
           \begin{pmatrix}
    \Vc{C}_{i,j}  & \Vc{C}_{i,j+1}    & \Vc{C}_{i,j+2}    &\Vc{C}_{i,j+3} \\
    \Vc{C}_{i,j}  & \Vc{C}_{i+1,j+1}   & \Vc{C}_{i+1,j+2}  & \Vc{C}_{i+1,j+3} \\
  \end{pmatrix} = \sum_k^{192}
           \begin{pmatrix}
             \Vc{A}_{i,k}  \\
             \Vc{A}_{i+1,k}   \\
           \end{pmatrix}  
           \begin{pmatrix}
             \Vc{B}_{k,j}  &  \Vc{B}_{k,j+1}  & \Vc{B}_{k,j+2}  &\Vc{B}_{k,j+3} \\
         \end{pmatrix}$} 
      \ENDFOR
    \ENDFOR
  \end{algorithmic}  
\end{algorithm}

\newpage 
\section{GEMM Tiles Generator}
In general, the core computation needs to be a multiple aligned to
$8$, say, for the channels (columns of the operands). Subvolumes being
a power of two are nice and not necessary. What if we can explore the
sub-volume solution space with some type of properties? In Equation
\ref{eq:ratio}, the ratio $\F$ is a function of the subvolume
($\F(m,n,k)$). This sub-volume can be for a core (streaming) or the
representation of the cluster computation (all cores with all
memtiles). We argue for three requirements: we should have a subvolume
that fit into a core (basic subvolume), we want to have largest number
of operations per subvolume ($\C(m,n,k)$), and among them we prefer
the one that have the largest ratio $\F = \frac{\C}{\ST}$ (where $\ST$
stands for space ans 24 KB represent the space for 6 banks in a
core). We can call $\T$ the set of valid solutions and we call $\W$
the sorted solution space as follows:

\begin{equation}
  \T = \{ [m,n,k]\mbox{ s.t. }\ST(m,n,k)< 24 KB \}
\end{equation}

\begin{equation}
  \W =  sort(\T, \lambda x: (-\C(x), -\F(x) )
\end{equation}
 

We call $\QQ_0$ the sum over $k$ of Equation \ref{eq:gemmA} and
$\QQ_1$ of Equation \ref{eq:gemmB}. These represent two algorithms and
also subproblems. The cluster of cores work together reading from
Mem-Tile and in the context of the larger problem we can double buffer
either $\Vc{A}$ or $\Vc{B}$.

If we chose $(m,n,k) = \W[0]$ as starting block for the core
computation. Then $\QQ_0$ will have a memtile subvolume
$M_0=(2*m,4*n,K)$ and $\QQ_1$ $M_1=(4*m,2*n,K)$. Both computations
will be the reduction of $K/k$ iterations. And we can choose the
algorithm so that $\F(M_0)$ and $\F(M_1)$ is the largest. The core
subvolume is determined by space and $\F$, so it is the Mem-tile
subvolume and thus the algorithm itself.

At this time, we will not deal with padding and alignment per core and
per memtile. We summarily say that the core subvolume $(m,n,k)$ will
determine a minimum zero padding so that a core will produce the
correct result. Padding at cluster level does not need to be, we can
accept either the turn off or columns/rows or the computation of
garbage and not writing back to DDR.

\subsection{Double Buffering aka DB}  

Rarely $\Vc{A}$ and $\Vc{B}$ will fit the Memtile and some double
buffering may be required. Given a core subvolume, If we can do double
buffering for both operands, there is no problem.


$\QQ_1$ is more efficient for large $\Vc{B}$, we tend to reuse
$\Vc{A}$, and at a minimum we need to double buffer $\Vc{B}$. Clearly
the core subvolume is related to the memtile subvolume and whether we
can double buffer. If we must tile the $K$ dimension, we can design
the algorithm and estimate the execution time. Then we can restart the
search space asking for a problem size that we can do DB of $\Vc{B}$
and compare.

$\QQ_0$ is more efficient for large $\Vc{A}$, we tend to reuse
$\Vc{B}$, and at a minimum we need to double buffer $\Vc{A}$. Clearly
the core subvolume is related to the memtile subvolume and whether we
can double buffer. If we must tile the $K$ dimension, we can design
the algorithm and estimate the execution time. Then we can restart the
search space asking for a problem size that we can do DB of $\Vc{A}$
and compare.

This search is almost complete. This is simply a set of heuristics for
the selection of the solution space, the algorithms, and the tiling.
In the, following section we shall present the ideas for a automatic
selection and exploration of the solution space and its optimal solution.




\section{Solution space and its exploration}
We are working on a software tool that will explore a solution space
for a given problem. For GEMM this is triplet $(M,N,K)$ (i.e., $\Vc{C}
\in M \times N, \Vc{A} \in M \times K, \Vc{B} \in K \times N$). Briefly,
  tiling $(M,N,K)$ to memtile is another triplet $(M_m,N_m,K_m)$. This
  can describe the (largest) problem we solve using data in
  memtile. We can go down further and describe the tiling at core
  level as $(M_c,N_c,K_c)$. In practice, $(M,N,K)=P$,
  $(M_m,N_m,K_m)=m$, and $(M_c,N_c,K_c)=c$ are enough to describe a
  complete computation.

The solution space is the composition of all valid ${\cal S}= \{
(P,m,c) \mbox{ s.t. valid}\}$. A tiling tools have two main goals:
First, we need to describe a complete set of solutions. In general,
this can be quite large; however hardware requirements and designs
strategies make the solution space a limited one. Second, we can address
cost functions and their effect on the overall solution space simply
by translating the solution $P,m,c$ either into actual code or other
realistic and consistent measures.

In all previous work, we did compute algorithm choices (such as
tiling) by back of the envelop cost estimates, code generation cost
estimate, simulations, and actual time measures. Here, we reduce to a
minimum the description of the algorithm and hardware. We compute
tiling for the application of double buffering at core and memtile
level. We aim at symmetric computations (so that we can have uniform
and predictable data movement).

In these subsections, we introduce the main idea for tiling and
solution space we use for each computation. We assume in all cases
that the computation start from DDR and we end in DDR. Problem $P$
will start in DDR and the result will be in DDR. The operand memory
allocation in DDR is given and the memtile is considered clean and
clear. If the P fit memtile, then $(P,P,*)$ may be a valid
solution. For this tool, we do not use addresses and we assume that
allocation is always possible if space is available. We are aware this
last statement is not always true.

\subsection{GEMM}
The current architecture is $4\times 2$ architecture and thus the
computation is rectangular by nature. We have two main algorithms
Equation \ref{eq:gemmA} and \ref{eq:gemmB}. We search first valid core
solutions $(M_c,N_c,K_c)=c$ using property of alignment specific to
the HW, then to the space requirements for the operands (with double
buffering).

For Equation \ref{eq:gemmA}, we need to read from memtile
$(2*M_c,4*N_c,K)=m$ and for \ref{eq:gemmB} we need
$(4*M_c,2*N_c,K)=m$. If K is too large, we will split K accordingly
($(4*M_c,2*N_c,K/4)=m$, say) and stream the computation.

First, the data in memtile is split perfectly across the cores. We aim
at padding at core level and if we can avoid better.  If we can double
buffer A or B or both we try, if we split by K, we double buffer
portions of A and B on the K dimension. Second we can split the
computation between DDR and memtile $m$ and thus we have the final
solution $P,m,c$. This describes completely our solution space from
core to original problem size.  We use a cost function, time, space,
compute and ratio to explore the solution space and the Pareto curve.

We provide two different classes and approaches: we provide a solution
space stemming from a single problem and from a set. A single problem
create a list of solutions but we do not take advantage of any
repetition. For a list of problem we create dictionary of
solutions. This can help us to explore solutions that would not be
available and consider in case extra constraints are introduced.


\subsection{MHA}
To exploit performance is to read into cores the matrices $\Vc{K}$ and
$\Vc{V}$ so that to perform the largest block inner product
$\sum_j\exp(\Vc{Q}_0\Vc{K}_j)*\Vc{V}_j$. The idea is to compute the
space requirement in order to compute locally in a core the numerator
Equation \ref{eq:sum0} and the denominator Equation
\ref{eq:sum1}. Then do a reduction by a single core for the final
computation Equation \ref{eq:resall}.

The minimum number of cores is one, if it is possible we rather use a
single column. The reduction is simpler done by a single column, and
the next column can work on $\Vc{Q}_1$ and $\Vc{R}_1$.

We start with a problem of size $P=(d,L_0,L_1,r)$ so that $\Vc{Q} \in
L_0\times d$, $\Vc{K} \in d\times L_1$, and $\Vc{V} \in L_1\times
r$. We find all the problem $c=(d,l_c,\ell_c,r)$ that fit in a core
and we can estimate the execution time.

\subsection{Layer Norm}
The solution space and cost estimation follow the same ideas developed
for MHA in the previous sections.

\subsection{Convolution}
The way we build the solution space is by observing the output tensor
from the smallest to the largest and compute the projection to the
input. Projection is a function that given an output tensor of size
$(1,H,W,C_{out})$ and a convolution with stride $(0, s_h, s_w,0)$ and
kernel $(C_{out}, k_h, k_w,C_{in})$, the projection is $(1,
(H-1)s_h+k_h, (W-1)s_w+k_w,C_{in})$. We have an starting problem ${\bf
  P}$.

Consider now that the output tensor will be created by two (or more)
columns of cores. We consider the computation split by width so that
each column compute may compute any $v \in [0,w/2]$ and each core
compute $c_o \in [0,C_{out}/4]$ with a proper granularity
(alignment). We assume we are streaming the computation by H (this is
important for the computation description and its time estimates)

The core will use a space for ${\bf c}$
\begin{itemize}
  \item output $(1,h,v,c_o)$ 
  \item input  $(1,(h-1)s_h*k_h,(v-1)s_w+k+w,C_{in})$ 
  \item weights $(c,k_h,k_w,C_{in})$ + $(1,1,1,c_o)$ 
\end{itemize}

For each core subvolume, we consider in memtile only multiple of the
core subvolume ${\bf m}$; that is, $H = m*h$ and $V=n*v$.
\begin{itemize}
  \item output $(1,H,V,c_o)$ 
  \item input  $(1,(H-1)s_h*k_h,(H-1)s_w+k+w,C_{in})$ 
  \item weights $(c,k_h,k_w,C_{in})$ + $(1,1,1,c_o)$ 
\end{itemize}



So for the same ${\bf c}$ we may have
multiple ${\bf m}$ that can fit in memtile with double buffering.

We have then a complete representation of the solution by our usual
triplet $({\bf P, m ,c})$ and we can estimate the performance by a
simple cost function. 

Notice, If we want to split the computation by input channel,
$C_{in}$, we may just transform the convolution into two or more
convolutions followed by element wise additions. If we want to split
the computation by $C_{out}$, we can transform the convolution into
two convolutions and a concatenation.

























%%%%%%%%% -- BIB STYLE AND FILE -- %%%%%%%%
\bibliographystyle{IEEEtran} \bibliography{ref}
%%%%%%%%%%%%%%%%%%%%%%%%%%%%%%%%%%%%

%\appendix{Review and Response}
%\input{review.tex}
\end{document}
