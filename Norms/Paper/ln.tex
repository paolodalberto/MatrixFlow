%% This is file `sample-manuscript.tex', generated with the docstrip
%% utility.  The original source files were: samples.dtx (with
%% options: `manuscript') IMPORTANT NOTICE: For the copyright see the
%% source file.  Any modified versions of this file must be renamed
%% with new filenames distinct from sample-manuscript.tex.  For
%% distribution of the original source see the terms for copying and
%% modification in the file samples.dtx.  This generated file may be
%% distributed as long as the original source files, as listed above,
%% are part of the same distribution. (The sources need not
%% necessarily be in the same archive or directory.)  The first
%% command in your LaTeX source must be the \documentclass command.
\documentclass[acmsmall]{acmart}
%\usepackage{bm}

%% as manuscript for review this will create line number 
\usepackage{lineno}
\usepackage{algorithm}
\usepackage{algorithmic}
%\linenumbers

%%  Always include hyperref last
%% \usepackage[bookmarks=true,breaklinks=true,letterpaper=true,colorlinks,linkcolor=black,citecolor=blue,urlcolor=black]{hyperref}

% Paolo's definition. Yes, I have my way to define stuff. Please take
% a look

\def\firmopsa{d}
\def\firmopsb{d}
\newcommand\AssignTags{TagIt}


% Definitions
% -----------
\def\x{{\mathbf x}}
\def\L{{\cal L}}
\def\spiral{SPIRAL\xspace}

\newcommand{\pred}{{}}

\newcommand{\tensor}[0]{\otimes}
\newcommand{\inv}[1]{{1{/}{#1}}}
%\newcommand{\inv}[1]{$\textstyle\frac{1}{#1}$}
\newcommand{\DFT}{\operatorname{\bf DFT}}
\newcommand{\WHT}{\operatorname{\bf WHT}}
\newcommand{\FIR}{\operatorname{\bf FIR}}
\newcommand{\one}[0]{{I}}
\newcommand{\eq}[0]{{=}}
\newcommand{\Tensor}[2]{#1{\tensor}#2}
\newcommand{\SPw}[2]{{\prec}#1,#2{\succ}}
\newcommand{\SPr}[2]{{<}#1,#2{>}}

\newcommand{\WTensorI}[2]{\Tensor{\WHT_{#1}}{\one_{#2}}}
\newcommand{\ITensorW}[2]{\Tensor{\one_{#1}}{\WHT_{#2}}}
\newcommand{\Wht}[2]{(\WTensorI{\WHT_{#1}}{#2})(\ITensorW{#1}{#2})}

\newcommand{\Times}[2]{{#1}{\times}{#2}}
\newcommand{\EQ}{{=}}
\newcommand{\Space}{{ S}}
\newcommand{\PEQ}{{{+}{=}}}
\newcommand{\SEQ}{{{-}{=}}}
\newcommand{\h}{\psi}
\newcommand{\balpha}{\bm{\alpha}}
\newcommand{\bgamma}{\bm{\gamma}}
\newcommand{\bbeta}{\bm{\beta}}
\newcommand{\bmu}{\bm{\mu}}
\newcommand{\bsigma}{\bm{\sigma}}


\newcommand{\f}{\varphi}
\newcommand{\Z}{\mathbb{Z}}
\newcommand{\A}{\mathbb{A}}
\newcommand{\B}{{\bf B}}
\newcommand{\X}{\mathbb{X}}
\newcommand{\T}{\mathbb{T}}
%\newcommand{\Q}{\mathbb{Q}}
\newcommand{\N}{\mathbb{N}}
\newcommand{\R}{\mathbb{R}}
\newcommand{\CC}{{\Gamma}}
\newcommand{\D}{\mathbb{D}}
\newcommand{\M}{\mathbb{M}}
\newcommand{\F}{\mathbb{F}}
\newcommand{\ST}{\mathbb{S}}
\newcommand{\Vc}[1]{{\boldsymbol #1}}
\newcommand{\fl}[1]{{\lfloor {#1} \rfloor}}
\newcommand{\cl}[1]{{\lceil {#1} \rceil}}
\newcommand{\half}[1]{{\frac{#1}{2}}}
\newcommand{\q}[1]{\Times{\cl{#1}}{\cl{#1}}}
\newcommand{\qq}[1]{\Times{\cl{#1}}{\fl{#1}}}
\newcommand{\qqq}[1]{\Times{\fl{#1}}{\cl{#1}}}
\newcommand{\qqqq}[1]{\Times{\fl{#1}}{\fl{#1}}}
\newcommand{\size}[1]{\sigma(#1)}
\newcommand{\Size}[3]{\sigma(\Vc{#1})\EQ\Times{#2}{#3}}
\newcommand{\n}{(\frac{n}{2})}
\newcommand{\m}{(\frac{n-1}{2})}
\newcommand{\Ceil}[1]{{\cl{\half{#1}}}}
\newcommand{\Floor}[1]{{\fl{\half{#1}}}}

\newcommand{\Endash}{{--}}
\newcommand{\Emdashbegin}{{---}}
\newcommand{\Emdashend}{{---}~}

\newcommand{\Transducer}[1] {{\mathbf #1 }}


\newcommand{\AO}  {\Times{\Ceil{m}}{\Ceil{n}}}
\newcommand{\AI} {\Times{\Ceil{m}}{\Floor{n}}}
\newcommand{\AII}{\Times{\Floor{m}}{\Ceil{n}}}
\newcommand{\AIII} {\Times{\Floor{m}}{\Floor{n}}}
\newcommand{\BO}  {\Times{\Ceil{n}}{\Ceil{p}}}
\newcommand{\BI} {\Times{\Ceil{n}}{\Floor{p}}}
\newcommand{\BII}{\Times{\Floor{n}}{\Ceil{p}}}
\newcommand{\BIII} {\Times{\Floor{n}}{\Floor{p}}}
\newcommand{\CO}  {\Times{\Ceil{m}}{\Ceil{p}}}
\newcommand{\CI} {\Times{\Ceil{m}}{\Floor{p}}}
\newcommand{\CII}{\Times{\Floor{m}}{\Ceil{p}}}
\newcommand{\CIII} {\Times{\Floor{m}}{\Floor{p}}}

\newcommand{\I}[2]{\Times{\Ceil{#1}}{\Ceil{#2}}}
\newcommand{\II}[2]{\Times{\Ceil{#1}}{\Floor{#2}}}
\newcommand{\III}[2]{\Times{\Floor{#1}}{\Ceil{#2}}}
\newcommand{\IV}[2]{\Times{\Floor{#1}}{\Floor{#2}}}


\newcommand{\Q}[2]{\Vc{#1}_{#2}}

\newcommand{\QAO}{{\Vc{A}_{0}}}
\newcommand{\QAI}{\Vc{A}_{1}}
\newcommand{\QAII}{\Vc{A}_{2}}
\newcommand{\QAIII} {\Vc{A}_{3}}

\newcommand{\QBO}  {\Vc{B}_{0}}
\newcommand{\QBI} {\Vc{B}_{1}}
\newcommand{\QBII}{\Vc{B}_{2}}
\newcommand{\QBIII} {\Vc{B}_{3}}

\newcommand{\QCO}  {\Vc{C}_{0}}
\newcommand{\QCI} {\Vc{C}_{1}}
\newcommand{\QCII}{\Vc{C}_{2}}
\newcommand{\QCIII} {\Vc{C}_{3}}
\newcommand{\FIGsize}[2]{{width={#1}, height={#2}}}

\newcommand{\Mod}[2]{{#1}{\text{ \bf mod }}{#2}}

\newcommand{\TableRef}[1]{Table \ref{#1}}
\newcommand{\FigureRef}[1]{Figure \ref{#1}}
\newcommand{\SectionRef}[1]{Section \ref{#1}}
\newcommand{\EquationRef}[1]{Equation \ref{#1}}
\newcommand{\OnPageRef}[1]{on page \pageref{#1}}

% Formatting
% ----------
%\setlength{\evensidemargin}{0mm}
%\setlength{\oddsidemargin}{0mm}
%\setlength{\textwidth}{6.5in}
%\setlength{\textheight}{9.5in}
%\setlength{\topmargin}{0in}
%\setlength{\headheight}{0in}
%\setlength{\headsep}{0mm}

\newenvironment{noinds_itemize}{\begin{list}{$\bullet$}
{\setlength{\rightmargin}{0em}
\setlength{\leftmargin}{1.2em}
\setlength{\itemsep}{0em}
\setlength{\topsep}{0em}
\setlength{\parsep}{0em}}}{\end{list}}



\newcommand{\mypar}[1]{{\bf #1.}}
\newcommand{\mytag}[2]{[#1]_{#2}}
\sloppy

%% jan 01, 2009 - dasdan - removed 'final' pdflatex cannot work with it.
%\usepackage[final=true,bookmarks=true,bookmarkstype=toc,colorlinks=true, linkcolor=blue,citecolor=blue]{hyperref}
%\usepackage[bookmarks=true,bookmarkstype=toc,colorlinks=true, linkcolor=blue,citecolor=blue]{hyperref}

% width fraction, mflops, relative, caption, label
\newcommand{\doublefigure}[5]{{\begin{figure}[htb]%
\centering %
\includegraphics[width=#1\linewidth]{#2}
\includegraphics[width=#1\linewidth]{#3}
\caption{#4}%bio                 
\label{#5}%
\end{figure}}}

\newcommand{\Doublefigure}[5]{{\begin{figure*} %
\centering %
\includegraphics[width=#1\linewidth]{#2}%
\includegraphics[width=#1\linewidth]{#3}%
\caption{#4}%
\label{#5}%
\end{figure*}}}

\newcommand{\Singlefigure}[4]{{\begin{figure*} %
\centering %
\includegraphics[width=#1\linewidth]{#2}%\figurebox{#1\linewidth}{}{}[#2]
\caption{#3}%
\label{#4}%
\end{figure*}}}

\newcommand{\singlefigure}[4]{{\begin{figure}[htb] %
\centering %
\includegraphics[width=#1\linewidth]{#2}%
%\includegraphics[height=#1\linewidth,angle=-90]{#3} %
%\figurebox{#1\linewidth}{}{}[#2]\\
\caption{#3}%
\label{#4}%
\end{figure}}}

\newcommand{\singlefigurerotate}[4]{{\begin{figure}[htb] %
\centering %
\includegraphics[width=#1\linewidth,angle=-90]{#2}%
%\includegraphics[height=#1\linewidth,angle=-90]{#3} %
%\figurebox{#1\linewidth}{}{}[#2]\\
\caption{#3}%
\label{#4}%
\end{figure}}}

\newcommand{\orthogonal}{%
\mathrel{\raisebox{.1em}{%     
\reflectbox{\rotatebox[origin=c]{90}{$\models$}}}}}


%%
%% \BibTeX command to typeset BibTeX logo in the docs
\AtBeginDocument{%
  \providecommand\BibTeX{{%
    \normalfont B\kern-0.5em{\scshape i\kern-0.25em b}\kern-0.8em\TeX}}}

%% Rights management information.  This information is sent to you
%% when you complete the rights form.  These commands have SAMPLE
%% values in them; it is your responsibility as an author to replace
%% the commands and values with those provided to you when you
%% complete the rights form.
\setcopyright{acmcopyright}
\copyrightyear{2025}
\acmYear{2025}
\acmDOI{}

%% These commands are for a JOURNAL article.
%\acmJournal{PACM}
%\acmVolume{--}
%\acmNumber{--}
%\acmArticle{---}
%\acmMonth{3}


%% These commands are for a PROCEEDINGS abstract or paper.
%\acmConference[Woodstock '18]{Woodstock '18: ACM Symposium on Neural
%  Gaze Detection}{June 03--05, 2018}{Woodstock, NY}
%\acmBooktitle{Woodstock '18: ACM Symposium on Neural Gaze Detection,
%  June 03--05, 2018, Woodstock, NY}
%\acmPrice{15.00}
%\acmISBN{978-1-4503-XXXX-X/18/06}


%%
%% Submission ID.
%% Use this when submitting an article to a sponsored event. You'll
%% receive a unique submission ID from the organizers
%% of the event, and this ID should be used as the parameter to this command.
%%\acmSubmissionID{123-A56-BU3}

%%
%% The majority of ACM publications use numbered citations and
%% references.  The command \citestyle{authoryear} switches to the
%% "author year" style.
%%
%% If you are preparing content for an event
%% sponsored by ACM SIGGRAPH, you must use the "author year" style of
%% citations and references.
%% Uncommenting
%% the next command will enable that style.
%%\citestyle{acmauthoryear}

%%
%% end of the preamble, start of the body of the document source.
\begin{document}

%%%%%%%%%%%---SETME-----%%%%%%%%%%%%%
\title{Norms: A Unified Approach}

\author{Paolo D'Alberto}
\email{}
\affiliation{%
  \institution{AMD}
  \streetaddress{2100 Logic Dr}
  \city{San Jose}
  \state{California}
  \postcode{95124}
}

\renewcommand{\shortauthors}{D'Alberto et al.}

\begin{abstract}

  Layer Norms are important tools in today's AI systems.  Also, they
  are the little pesky layers that inference-engine tools have often
  problems in either fusing them with other layers for speed-up or
  because they introduce numerical unwanted oddities at low precision.
  Norms have linear complexity. That is, they are communication bound,
  they require to read the input at least twice, and there is a
  satisfying and contrasting asymmetry of the computation (i.e., row
  Gaussian normalization and column scaling).
  
  Here, we unify the norm formalism. We want to clarify the general
  computation, the opportunities for fusion, and numerical
  properties. So we can generate consistent and correct code for our
  AIE AMD computes.
  
\end{abstract}

\maketitle

\section{Definitions and computations } 
\label{sec:introduction}

A matrix is $\Vc{X} \in \R^{M \times N}$ with $M$ rows and $N$ columns
and we represent a single scalar element as $x_{i,j}$ with obvious
bounds. A vectors is $\Vc{V} \in \R^{M}$ with scalar element $v_i$.
Last, a scalar is $\alpha \in \R$.


A matrix norm is a two-part computation. We have a {\em projection}
where we compute a factor, a vector, and then a {\em normalization}
where we take the projected factor and distribute it back to the
elements of the matrix. In the literature and implementations, these
computations have also other names: Projection is a reduction because
it is frequently associated with sums, dot products. Normalization is
a broadcast. Now that we have a names, let us define them formally and
constructively.

Let us start with scalar operations on Matrix/Vectors:
\begin{definition}
  A scalar operation on a matrix $\Vc{X}$
  \begin{equation}
    G_f: \Vc{X} \rightarrow \Vc{Y} \text{ and }  y_{i,j} = f(x_{i,j})
  \end{equation}
\end{definition}

For example, we will use $G_{x*x}(\Vc{X}) = x_{i,j}^2$ and
$G_{\exp}(\Vc{X}) = \exp(x_{i,j})$. The definition goal is the
computation abstraction, this is also called a element wise operation.


\begin{definition}
An associative Projection $P_+: \Vc{X}\in\R^{M\times N} \rightarrow
\Vc{F} \in \R^{M}$, is a function where the element
\[ f_i = \sum_{j=1}^N x_{i,j}. \]
The $+/\sum$ sign has the general properties of classic addition and
the most important the associative property:
\[
f_i = (x_{i,1}) + (\sum_{j=2}^N x_{i,j}) = (\sum_{j=1}^{N-1} x_{i,j}) + (x_{i,N}).
\]
\end{definition}

This is a row projection but column projections are similar.  As we
will read our matrix, we are going to maintain the order and
projecting of matrix $\Vc{X}$ (row elements $x_{i,*}$). The
projections need not to be commutative in general but all the
projections of interests are. Also, we require the computation of
multiple projections for the computation of the normalization factor.

We abuse the notation of the symbol $+$ in the associative expression
$f_i = (x_{i,1} + (\sum_{j=2}^N x_{i,j}))$, in the sense that the
combination of the temporary sub-projection may require some extra
work during the computation or at the end when we have a complete
projection into a proper normalization factor. However, such
simplification does not detract any information and the implementation
can address it.

What could $P_+(G_f(X))$ be? That is, we apply scalar operations and
then projection. The associativity should be a property of the
projection and not of the value of the matrix. SoftMax Norm in finite
precision is an example where we have {\em to work} to enforce the
associativity (e.g., $\exp(x)$ can be large and in finite precision
overflow easily especially their sum).

\begin{definition}
  An optional summary function $S_+: \{\Vc{F}\}_i\in\R^{ M}
  \rightarrow \Vc{E} \in \R^{M}$ and  $\Vc{F}_i = P_i(G_{f_i}(\Vc{X}))$
\end{definition}
So we can summarize the first part of the computation by an
associative projection or a set of projections and an optional summary
function for the final composition of the normalization factors.

Now let start formulating the normalization computation.

\begin{definition}
  We have two normalization functions: first, by row
  \begin{equation}
    N_*: \Vc{X}\in\R^{M\times N} \times \Vc{T}\in \R^M \rightarrow \Vc{Y}
    \in \R^{M\times N}, \text{ where } y_{i,j} = x_{i,j}*t_i
  \end{equation}
  and second by column
  \begin{equation}
    N_*: \Vc{X}\in\R^{M\times N} \times  \Vc{S}\in \R^N \rightarrow \Vc{Y}
    \in \R^{M\times N}, \text{ where } y_{i,j} = x_{i,j}*s_j
  \end{equation}
  Again the $*$ is not necessarily a multiplication and the
  normalization can be by row or by column. For presentation purpose
  we will use a single symbol (unless confusing).
\end{definition}

If we extend the normalizing factor $\Vc{F}$ into a diagonal matrix in
$\Vc{D}_F \in \R^{M\times M}$ so that $[\Vc{D}_F]_{i,i} = f_i$ and
zero everywhere else (same idea for $\R^{N\times N}$), and the
normalization is a based on a scalar product, then we reduce the
computation to matrix-matrix multiplication products. 
\begin{equation}
  \Vc{D}_F * \Vc{X} \text{ and } \Vc{X} * \Vc{D}_F
\end{equation}

Notice that using matrix multiplication we can see that a row
normalization is associative $(\Vc{D}_f * \Vc{X}) * \Vc{Y} = \Vc{D}_f
* (\Vc{X} * \Vc{Y})$ then giving an hint how the normalization could
be postponed if neceesary.  Also a column mormalization in combination
with a matrix multiplication $(\Vc{X}* \Vc{D}_f) * \Vc{Y} = \Vc{X}*
(\Vc{D}_f * \Vc{Y})$ is actually a row normalization of the following
operand.

We introduce the last definiition, a matrix partition.
\begin{definition}
  A row partition of a matrix is simply defined as
  \begin{equation}
    [\Vc{X} ]_{r=2} = \
    \begin{pmatrix}
      \Vc{X}_1  \\
      \Vc{X}_2  \\
    \end{pmatrix}\\
  \end{equation}
  and a column partition
  \begin{equation}
    [\Vc{X} ]_{c=2} = \
    \begin{pmatrix}
      \Vc{X}_1  & \Vc{X}_2  \\
    \end{pmatrix}\\
  \end{equation}
\end{definition}
If we want to take a matrix $\Vc{X} \in \R^{M\times N}$ and normalize
its rows by a square norm $||x||_2$, see Algorithm \ref{alg:l2}. As we
define $P_+(\Vc{X})_i$ as the summation $\sum_j x_{i,j}$ and
$N_*(\Vc{X}, \Vc{T})_{i,j} = x_{i,j}*t_i$, we have the complete
computation of the norm : $f_i = \sqrt{\frac{1}{N}\sum_{j=1}^N
  x_{i,j}^2}$ and $x_{i,j} = \frac{x_{i,j}}{f_i}$.

\begin{algorithm}
    \caption{$\Vc{X}/||\Vc{X}||_2$ algorithm}
    \label{alg:l2}
    \begin{algorithmic}
      \STATE $\Vc{T} = P_+(G_{x*x}(\Vc{X}))$
      \STATE --------------------------------
      \STATE $\Vc{T} = \sqrt{\Vc{T}/N}$    \COMMENT{ \# Summary} 
      \STATE --------------------------------
      \STATE $\Vc{X} = N_*(\Vc{X},\Vc{T})$  
    \end{algorithmic}
\end{algorithm}

When we introduce a new projections $P_{max}(\Vc{X})_i =
\max{x_{i,*}}$, which is associative, and we normalize the additive
projection with the result of the new projection, we can see that the
Softmax computation has the same properties and composition (Algorithm \ref{alg:l2_2}).

\begin{algorithm}
    \caption{$SoftMax(\Vc{X})$ algorithm}
    \label{alg:l2_2}
    \begin{algorithmic}
      \STATE $\Vc{M} = P_{max}(\Vc{X})$
      \STATE $\Vc{T} = P_+(G_{exp}(\Vc{X})*G_{exp}(-\Vc{M}))$
      \STATE --------------------------------
      \STATE $\Vc{T} = \Vc{T}*\exp{\Vc{M}}$    \COMMENT{ \# Summary}
      \STATE --------------------------------
      \STATE $\Vc{X} = N_*(G_{exp}(\Vc{X}),\Vc{T})$ 
    \end{algorithmic}
\end{algorithm}


Let us go back using the original notations and block the input matrix
so that to exploit its natural composition, independent computations
by row and we exploit the associativity of the projection in Algorithm \ref{alg:l2_3}

\begin{algorithm}
    \caption{Blocked $3\times 3$}
    \label{alg:l2_3}
    \begin{algorithmic}
      \STATE \[
        \Vc{X}  =   \begin{pmatrix}
          \Vc{X}_{1,1} &    \Vc{X}_{1,2}&    \Vc{X}_{1,3}  \\
          \Vc{X}_{2,1} &    \Vc{X}_{2,2}&    \Vc{X}_{2,3}  \\
          \Vc{X}_{3,1} &    \Vc{X}_{3,2}&    \Vc{X}_{3,3}  \\
        \end{pmatrix}
      \] \\
      \WHILE{$i$ in [1,2,3] \# Independent } 
          \STATE $\Vc{T} = \Vc{0} $  \COMMENT{ \# Zero element for $P_+$}
          \WHILE {$j$ in [1, 2,3] \# Projection }
              \STATE $\Vc{T} = \Vc{T} +P_+(G_{f}(\Vc{X}_{i,j}))$ \COMMENT{   \# Reduction}  
          \ENDWHILE
          \STATE --------------------------------     
          \STATE $\Vc{T}  =   S_+(\Vc{T})$       \COMMENT{ \# Summary}
          \STATE --------------------------------     
          \WHILE {$j$ in [1,2,3] \# Normalization }
             \STATE $ \Vc{X}_{i,j} = N_*(\Vc{X}_{i,j},\Vc{T})$  
          \ENDWHILE
      \ENDWHILE
    \end{algorithmic}
\end{algorithm}

The application of the blocked algorithm for the Euclidean Norm and
for Softmax requires very little re-work, just remember that softmax
will require two projections:

\begin{center}
  \begin{tabular}{l}
    \hline
    SoftMax associative projections \\
    \hline 
    $\Vc{M}_0 = -\infty; \Vc{T}_0 = 0$ \\ 
    $\Vc{M}_{j+1} = max(M_j, P_{max}(\Vc{X}_{i,j})$ \\
    $\Vc{T}_{j+1} = \Vc{T}_j*(exp(-(\Vc{M}_{j+1}-\Vc{M}_j))) + P_+(G_{\exp}(\Vc{X}_{i,j}-\Vc{M}_{j+1}))$ \\
  \end{tabular}
\end{center}

Once we recognizing the particular projection, the matching of the
proper computation should be natural. Also, the layer norm $(x_{i,j} -
\mu_i)/\sqrt{\sigma^2_i}$ will require the computation of two
projections so that average ($\mu$) and variance ($\sigma^2$) can be
computed in a single pass but with two projections (Algorithm
\ref{alg:l2_4}).

\begin{algorithm}
    \caption{Layer Norm $3\times 3$ with $\gamma, \beta$}
    \label{alg:l2_4}
    \begin{algorithmic}
      \STATE \[
        \Vc{X}  =   \begin{pmatrix}
          \Vc{X}_{1,1} &    \Vc{X}_{1,2}&    \Vc{X}_{1,3}  \\
          \Vc{X}_{2,1} &    \Vc{X}_{2,2}&    \Vc{X}_{2,3}  \\
          \Vc{X}_{3,1} &    \Vc{X}_{3,2}&    \Vc{X}_{3,3}  \\
        \end{pmatrix}
      \] 
      \STATE \[
        \Vc{\gamma}  =   \begin{pmatrix}
          \Vc{\gamma}_{1} &    \Vc{\gamma}_{2}&    \Vc{\gamma}_{3}  \\
        \end{pmatrix}
      \] 
      \STATE \[
        \Vc{\beta}  =   \begin{pmatrix}
          \Vc{\beta}_{1} &    \Vc{\beta}_{2}&    \Vc{\beta}_{3}  \\
        \end{pmatrix}
      \] 
      \WHILE{$i$ in [1,2,3] }
          \STATE $\Vc{T} = \Vc{0} $  
          \STATE $\Vc{S} = \Vc{0} $ 
          \WHILE {$j$ in [1, 2,3] }
              \STATE $\Vc{T} = \Vc{T} +P_+(\Vc{X}_{i,j})$             \COMMENT{Projection + sum}  
              \STATE $\Vc{S} = \Vc{T} +P_+(G_{x*x}(\Vc{X}_{i,j}))$ \COMMENT{Projection + sum squares}  
          \ENDWHILE
          \STATE --------------------------------     
          \STATE $\Vc{T}  =   \Vc{T}/N$       \COMMENT{Summary, scalar}
          \STATE $\Vc{S}  =   (\Vc{S} - (\frac{\Vc{T}}{\sqrt{N}})^2)/N$       \COMMENT{Summary, scalar}
          \STATE --------------------------------     
          \WHILE {$j$ in [1,2,3] }
             \STATE $ \Vc{X}_{i,j} = \Vc{X}_{i,j}-\Vc{T}$  \COMMENT{Normalization Row}
             \STATE $ \Vc{X}_{i,j} = \Vc{X}_{i,j}/\Vc{S}$   
             \STATE $ \Vc{X}_{i,j} = \Vc{X}_{i,j}*\gamma_j$   \COMMENT{Normalization Column}
             \STATE $ \Vc{X}_{i,j} = \Vc{X}_{i,j} + \beta_j$   
          \ENDWHILE
      \ENDWHILE
    \end{algorithmic}
\end{algorithm}


\section{Computation Modes} 
The mathematical notation provides a clear description of the
computation, clearly pedantic but constructive in nature. Also, the
normalization factor and its computation can be used when we would
like to fuse the norm operation with the following (or the previous)
computation.

Here, we are more interested in describing how we would implement the
norm computation using different engine: GPU, GPU, and AIE FGPA
engines. We show direct implementations for CPU and AIE and we present
a short discussion about the GPUs. This is always related to AMD chips
and codes.

The main difference is about the computation of the Projection and how
the hardware implicitly helps the computation. Let's start with the
GPU that uses heavily the associativity for parallelization of the
computation.

\subsection{GPU conversation}
The projection is associative and we can exploit the parallelism by
blocking the computation in warps, compute independent and separated
$\Vc{T}_i = P_+(\Vc{X}_{i,j})$, then reduce the computation in cores
by using a tree like computation and built in HW utilities (register
computations). 

\begin{equation}
  \begin{matrix}
    \Vc{T}_1 = P_+(\Vc{X}_{i,1}), & ... &  \Vc{T}_n = P_+(\Vc{X}_{i,n}) \\ 
    \Vc{T}_1 = \Vc{T}_1+\Vc{T}_2, & ... &  \Vc{T}_{\frac{n}{2}} = \Vc{T}_{\frac{n}{2}} + \Vc{T}_{\frac{n}{2}+1} \\
    \Vc{T}_1 = \Vc{T}_1 + \Vc{T}_2, & ... &  \Vc{T}_{\frac{n}{4}} = \Vc{T}_{\frac{n}{4}} + \Vc{T}_{\frac{n}{4}+1} \\
    \dots  & ... & ... \\ 
    \Vc{T}_1 = \Vc{T}_1 + \Vc{T}_2 & & \\
  \end{matrix}
\end{equation}

The small size of the computation that can be done as unit and the HW
support for the reduction makes this computation
appealing. Numerically, the binary three computation makes the
computation balanced and well partitioned. If there is no use of high
precision accumulators this will have numerical advantages. That is, a
balanced binary tree with $N$ elements has the longest addition path
length of only $K = \log_2(N)$ and the fish spine tree has $K = N-1$
and for additions we accumulate an error of $O(K)$.

For a Layer Norm of size $512 \times 768$ on a MI100 with 1.2 TB/s
HBM2 we can achieve 9 us (microseconds) latency (bandwidth utilization
800GBs) for layer norm in fp16 and fp32 precision. This is the fastest
performance across the AMD products at our disposal.

\subsection{CPU conversation}
CPUs have deep memory hierachies to exploit temporal locality. If we
observe the norm computations there is little temporal locality. Each
row is read twice, but there will be $N$ accesses in between. As soon
as we realize that $\Vc{T} += P_+(\Vc{X}_{i,j})$ has consecutive
accecesses to fast L1 caches we can exploit the AVX instruction set to
transform scalar operations into parallel vector operations.

The partition on $\Vc{X}$ will associate a set of row to a
thread/core. We stream the input rows and we store in the core the
temporary $T$ (registers), we compute the factor(s), then we stream
the rows again for the propagation of the normalization factor(s). We
exploit temporal locality because the smallest and fastest cache in
the memory hierarchy containing the rows will be read twice naturally
without any further assistance. When the computation moves to the next
rows, we clear the memory hierarchy safely. C threads can deploy this
strategy naturally and without the need to much careful
blocking. However, threading is heavy and they are really applicable
only for very large matrices.

{\tiny  \begin{verbatim}
Vendor ID:               AuthenticAMD
  Model name:            AMD EPYC 7F52 16-Core Processor
Caches (sum of all):     
  L1d:                   1 MiB (32 instances)
  L1i:                   1 MiB (32 instances)
  L2:                    16 MiB (32 instances)
  L3:                    512 MiB (32 instances)
\end{verbatim} }

By writing naive code for x86 ISA and using as reference the $512
\times 768$, the base code reference is about 600 us. By exploiting a
different style in exploiting parallel row computations and using the
AVX instructions (using {\em --fast-math}), then we can achieve 121 us
using a single core and float point 32bits.

For comparison to an older threadripper with share L3 cache 
{\tiny  \begin{verbatim}
Vendor ID:                AuthenticAMD
  Model name:             AMD Ryzen Threadripper 1950X 16-Core Processor
Caches (sum of all):      
  L1d:                    512 KiB (16 instances)
  L1i:                    1 MiB (16 instances)
  L2:                     8 MiB (16 instances)
  L3:                     32 MiB (4 instances)
\end{verbatim} }
we can achieve 269 us.

\section{AIE Core computation, Tiling generation, and validation}
As matter of fact, we use float 64bit precision in the code generation
presented in this repository. However, the AIE instruction set has
vector operations per core (like GPU), for int8, int16, bfloat16 and
float32. The AIE system has a three level memory hierarchy: L3
connected to four L2 (memtile one for each column) and each L2 is
connected with one column of four cores L1 and one specific row with
four core L1.

The reduction using GPU style codes is possible and encouraged for
each core internally but there is no simple equivalent for extra-core
reduction. We suggest a Norm computation on AIE similar to the CPU in
the sense that we use the memory hierarchy to stream row partitions
and accumulate in core the projections. Then, we do another pass. The
main difference that the data movement has to be orchestrated
explicitly because there is not hardware assisted data replacement
policy.

Let us rewrite the blocked algorithm here once again to show the
logical division of the computation but we want to draw attention to
the same inner loop reading the input. A reminder the temporary
projections, summary, and normalizing factors will be store in core
(L1). We will have a section related to the case when and where we
must spill (to L2).

{\small \begin{algorithm}
    \caption{Blocked $3\times 3$ (Projection in L1)}
    \label{alg:l2_5}
    \begin{algorithmic}
      \STATE \[
        \Vc{X}  =   \begin{pmatrix}
          \Vc{X}_{1,1} &    \Vc{X}_{1,2}&    \Vc{X}_{1,3}  \\
          \Vc{X}_{2,1} &    \Vc{X}_{2,2}&    \Vc{X}_{2,3}  \\
          \Vc{X}_{3,1} &    \Vc{X}_{3,2}&    \Vc{X}_{3,3}  \\
        \end{pmatrix}
      \] \\
      \WHILE{$i$ in [1,2,3] \# Independent } 
          \STATE $\Vc{T} = \Vc{0} $  \COMMENT{ \# Zero element for $P_+$}
          \WHILE {$j$ in [1,2,3] \# Projection }
              \STATE $\Vc{T} = \Vc{T} +P_+(G_{f}(\Vc{X}_{i,j}))$ \COMMENT{   \# Reduction}  
          \ENDWHILE
          \STATE --------------------------------     
          \STATE $\Vc{T}  =   S_+(\Vc{T})$       \COMMENT{ \# Summary}
          \STATE --------------------------------     
          \WHILE {$j$ in [1,2,3] \# Normalization }
             \STATE $ \Vc{X}_{i,j} = N_*(\Vc{X}_{i,j},\Vc{T})$  
          \ENDWHILE
      \ENDWHILE
    \end{algorithmic}
\end{algorithm} }


\subsection{Core computation}
Take the prospective of a core $C_i$ that, somehow, can access from a
stream of data a block $\Vc{X_{i,k}}$. In a CPU implementation, the
core starts the reading of the block by a load and the HW may go as
far the OS page to procure the data (page, DRAM Row, L3, L2, L1d). By
the AIE programming model, imagine we create a stream where
$\Vc{X_{i,1}}$ ... $\Vc{X_{i,N}}$ is send through and it is repeated
twice, we call the location where data arrives $L_1$.  The core just
need to wait for the block, release the block when done, and wait for
the next. Now by literally counting to $N$, the core knows when to do
the projection, the summary, and the normalization. The Algorithm
\ref{alg:l2_6} provides a clear boundary of the core computation as a
kernel and how the functionality is based on counting the number of
partitions read and when to write.


{\small \begin{algorithm}
    \caption{Blocked $3\times 3$}
    \label{alg:l2_6}
    \begin{algorithmic}
      \WHILE{$i$ in [1,2,3] \# Independent } 
          
          \WHILE {$r$ in [1,2] followed $j$ in [1, 2, 3] }
             \STATE \COMMENT{ \# Core Computation} 
             \IF{$r*j==1$ }
             \STATE $\Vc{T} = \Vc{0} $  \COMMENT{ \# Zero element for $P_+$}
             \ELSIF{$r ==1$ }
             \STATE $\Vc{T} = \Vc{T} +P_+(G_{f}(\Vc{X}_{i,j}))$  
             \ELSIF {$r==2$  and $j==1$}
             \STATE $\Vc{T}  =   S_+(\Vc{T})$      
             \ELSE
             \STATE $ \Vc{X}_{i,j} = N_*(\Vc{X}_{i,j},\Vc{T})$
             \ENDIF
         \ENDWHILE
      \ENDWHILE
    \end{algorithmic}
\end{algorithm} }

Using a common terminology for the AIE programming: $r$ is the
repetition, $\Vc{X}$ is the buffer, $\Vc{X}_{i,k}$ is the tile, and
$\Vc{X_{i,1}}$ ... $\Vc{X_{i,N}}$ is the (time) traversal of the
buffer by tiles with the specified repetition. In this particular
norm, Projection and Reduction require the same memory foot print
because they both read $\Vc{X_{i,N}}$ and only one write
$\Vc{X_{i,N}}$. If we introduce column normalization such as in
Algorithm \ref{alg:l2_4}, the normalization pass will require to read
$\Vc{X_{i,k}}$, $\gamma_k$, and $\beta_k$. We will address space
constraints and tiling in the next section.

\subsection{Tiling}
\label{sec:tiling}
\singlefigure{0.50}{AIE.png}{4x4 AIE representation}{fig:aie}

Let us start with our interpretation of tiling by using an abstract
representation of the memory hierarchy and connections as in Figure
\ref{fig:aie}: L3 = DDR = infinity, L2 is composed of four mem-tiles
of 512KB, and each mem-tile has a connection to one column of cores
and one different connection to a row. Here, we assume that the
connections are broadcasts: the same data is broadcast to all cores
and the core will select a subpart (if it likes). Each core has eight
banks of 8KB each composing the lowest level L1, we give two banks for
inputs for ping pong, two for weights, two for outputs, and two banks
for Temporary Space. 

Intuitively, Tiling is a {\em spatial} and a {\em temporal}
partition. Where, a partition describes a way to take a matrix and
split into (non-overlapping) parts in order to cover the whole
original matrix and in principle move it. We assume a row-major
layout.

\begin{definition}
   Consider a matrix $\Vc{X} \in
  \R^{M\times N}$, 
  \begin{equation}
    \Vc{X}  = \
    \begin{pmatrix}
      \Vc{X}^1  \\
      \Vc{X}^2  \\
      \Vc{X}^3  \\
      \Vc{X}^4  \\
    \end{pmatrix} = \dot\sum_i^r \Vc{X}^i
  \end{equation}
  a row partition is easily specified by an offset address
  Address($\Vc{X}_i$) = $(i-1)\frac{M}{4}N$ and the tiles are in
  contiguous memory addresses. In similar fashion, in a column
  partition
  \begin{equation}
    \Vc{X}  = \
    \begin{pmatrix}
      \Vc{X}^1  & \Vc{X}^2  &  \Vc{X}^3  &   \Vc{X}^4  \\
    \end{pmatrix} = \dot\sum_i^c \Vc{X}^i
  \end{equation}
  the offset addresses are Address($\Vc{X}_i$) = $(i-1)\frac{N}{4}$
  and the tiles are stored in a strided spaces.  These represent {\bf
    spatial partitions}, each partition has a buffer $\Vc{X}^i$ and
  each is independent.
\end{definition}


\begin{definition}
  Consider a buffer $\Vc{X}^k$ and partition it into tiles
  $\Vc{X}_{i,j}$ and the {\bf temporal partition} $\Vc{X}^k =
  \sum_{i,j}^{rc}\Vc{X}^k_{i,j}$ (by row and then columns). Think of a
  stream where $\Vc{X}^k_{i,j}$ is streamed in order so that the whole
  buffer eventually has been touched/transfer. There is
  $\sum_{i}^r\Vc{X}^k_{i}$ by row and $\sum_{i}^c\Vc{X}^k_{i}$ by
  column as a temporal partition.
\end{definition}

Clearly, we are abusing the $\sum$ symbol. In this contest, the symbol
represents an iteration space that our computation needs to split in
space and time. Note, $\dot\sum$ is a way to split the operand and
thus the computation spatially.  $\dot\sum^c$ we split spatially and
vertically and $\dot\sum^r$ we split spatially and horizontally.
Thus, $\sum$ is a way to split the computation temporally: $\sum^c$ we
split temporally and vertically and $\sum^r$ we split temporally and
horizontally. We can combine these operators constructively and
intuitively as it follows.

\begin{definition}
  The L3 to L2 traversal of a matrix $\Vc{X}$ is the composition of a
  spatial and temporal partition: For example, $T(\Vc{X}) =
  \dot\sum_k^r \Vc{X}^k = \dot\sum_k^r\sum_{i}^{r}\Vc{X}^k_{i}$.
  $\Vc{X}$ is the original buffer, there are four spatial tiles
  $\Vc{X}^k$, each is the buffer that will be streamed by temporal
  partition $\sum_{i}^r\Vc{X}^k_{i}$ with tile $\Vc{X}^k_{i}$ (by row,
  by column, and both)
\end{definition}

\subsubsection{Tiling LLM}
A L3 to L2 traversal is valid if the tile size fits L2 (without
context and without double buffering). More constraints are introduced
when more information are available. For Norms we need double
buffering and we stream at very least $\Vc{X}$ and the output of the
Norm $\Vc{Y}$. In general, a complete tiling/traversal for L3, L2, and
L1 could be summarized using our notation by loops: spatial by row,
temporal by row, spatial by row, temporal by row (all independent
computations).
\begin{equation}
T(\Vc{X}) = \dot\sum_k^r \Vc{X}^k =
\dot\sum_k^r\sum_{i}^{r}\Vc{X}^k_{i} =
\dot\sum_k^r\sum_{i}^{r}\dot\sum_l^r\Vc{X}^{k,l}_{i} =
\dot\sum_k^r\sum_{i}^r\dot\sum_l^r\sum_{m}^r\Vc{X}^{k,l}_{i,m}.
\end{equation}

Note, the second spatial partition is a way to take the matrix in L2,
split it so that it will feed each core L1 separately, but because we
use broadcast, there is not second spatial partition and all core
share the same data:
\begin{equation}
\dot\sum_k^r\sum_{i}^r\sum_{m}^r\Vc{X}^{k}_{i,m}.
\end{equation}

\begin{itemize}
\item If we observe just the tiling and the first partition using
  split by column, this split marks the beginning of temporal reuse,
  the computation will come back for the normalization and this split
  will be {\em repeated} twice. This computation will require blocked
  reduction.

\item Using our notation, the following tiling formulations imply
  formal reuse in L1 (we read L1 twice)
  \begin{equation}
    \begin{matrix}
    \dot\sum_k^r\sum_{i}^r\sum_{m}^r\Vc{X}^{k}_{i,m}, & 
    \dot\sum_k^r\sum_{i}^r\sum_{m}^C\Vc{X}^{k}_{i,m}, & 
    \dot\sum_k^r\sum_{i}^r\sum_{m}^{RC}\Vc{X}^{k}_{i,m}. \\
    \end{matrix}
  \end{equation}
  
\item These imply reuse in L2
  \begin{equation}
    \begin{matrix}
    \dot\sum_k^r\sum_{i}^{C}\sum_{m}^c\Vc{X}^{k}_{i,m}, & 
    \dot\sum_k^r\sum_{i}^{RC}\sum_{m}^c\Vc{X}^{k}_{i,m}. \\
    \end{matrix}
  \end{equation}

\item This implies reuse in L3
  \begin{equation}
    \dot\sum_k^{C}\sum_{i}^c\sum_{m}^c\Vc{X}^{k}_{i,m}.
  \end{equation}

\item This implies two separate passes and spill to L2/L3 of
  $P_+(\Vc{X})$ for which. The first split is a temporal split
  repeating the overall process twice (this is not possible in MLADF
  but we can split the main computation into smaller ones).
  \begin{equation}
    \sum_\ell^{r}\dot\sum_k^{C}\sum_{i}^c\sum_{m}^c\Vc{X}^{k}_{\ell,i,m}.
  \end{equation}
\end{itemize}

The AIE traversal implementation is more like the notation above. That
is, the inner loops describe only one sub partition simplifying the
implementation and enforcing a partition that has same shapes. Also
some repetition are not possible 

However, the logical partitioning of any matrix is a tool completely
general and with some experience, is applicable for different levels
and connections. We use it for the representation of bilinear matrix
multiplication algorithms in the past.

In practice, we implement Tiling as recursive function, and the data
structure is like a tree once built and unfolded.
\begin{equation}
  T[L3\rightarrow L2](X) = \dot\sum_k^r\sum_{i}^rT[L2\rightarrow L1](\Vc{X}^{k}_{i}).
\end{equation}

\subsubsection{Tiling Unified}
The unified overlay use the connection between DDR and L2 memtiles in
a different pattern.

We still consider DDR infinite. From the DDR, First, the inputs and
the outputs will be able to access a logical L2 memory (IOL2) composed
of up to 3 memtiles, thus 512*3*KB and memtile M1, M2, and M3. Second
the weights will access a separate and single memtile M4 (WL2).  The
logical IOL2 has 4 vertical connections one for each column to each
core. Each column communication is a broadcast. The WL2 has 4 row
connection, and each row connection is a broadcast. Each core in each
column produces a separate output and it has is custom channel back to
IOL2.

The tiling in this architecture follows this rule: temporal, spatial,
temporal.

\begin{equation}
T(\Vc{X}) = \sum_k^r \Vc{X}_k =
\sum_k^r\dot\sum_{i}^{r}\Vc{X}^i_{k} =
\sum_k^r\dot\sum_{i}^{r}\sum_l^r\Vc{X}_{k,l}^{i} =
\end{equation}



\subsection{Tiling generation}
\label{sec:tiling-generation}
There are always computation and hardware constraints that enforce
granularity in the shape and size of the tiles. At this time we do not
aim to an optimal tiling but we explore and present constraints-based
heuristics. Two main ideas are applied: a top down partition and
double buffering is preferred. We have a fit function, see Algorithm
\ref{alg:fit}.

{\small
  \begin{algorithm}
    \caption{Fit : Tiling, L, SplitFunction, multiplicity, granularity }
    \label{alg:fit}
    \begin{algorithmic}
      \STATE Components = Tiling.spatial-parts    \# list of matrices 
      \WHILE{ c in Components}
        \STATE T  = None;  t = 1;         q = True
        \WHILE{ q is True}
        \IF {t*gran is too large}   
            \STATE q = False; continue
        \ELSIF {t*gran has reminder }  
            \STATE continue
        \ENDIF
        \STATE t = SplitFunction(c,t*gran) 
        \STATE A = t[0]   \COMMENT{ \# Assume the first partition is the largest}
        \IF{multiplicity*A.space()>L}
           \STATE q = False
        \ELSE 
           \STATE T = A
        \ENDIF
        \STATE t+=1
        \ENDWHILE
      \STATE c = Tiling(c, T) 
      \ENDWHILE
    \end{algorithmic}
  \end{algorithm}
}

We choose the largest partition for which the tiles size fit the level
capacity L with or without double buffering (multiplicity=1 or
2). Making sure that the size of the tile has a proper granularity is
to make sure that the following tiles may have the same granularity
satisfied without awkward reminder making the index computation and
problem size not properly balanced.

In general, we start by trying splitting the computation by row
$\dot\sum_k^r\sum_{i}^r T(\Vc{X}^{k}_{i})$. We fit recursively the
Tiling. If it does not fit, then we try by column
$\dot\sum_k^c\sum_{i}^c T(\Vc{X}^{k}_{i})$ and we know that the rest
will be by column.


What is the implication of the latter case, when we start splitting
temporally by columns? We split the matrix in space by row in L3 into
4 parts. Each spatial partition will transfer at least a complete
column, this will be transfer to L2 as it is, and to L1 (broad cast).
L1 is 8 KB of data, if we assume that each element is two Bytes, L1
and double buffering can hold 4*K elements, if we have a single
projection, we are good to go and we can handle a $M=16K$ elements.
For layer norm, we need to keep two projections and thus the maximum
size is 2*K read elements and thus ($M = 8K$). For even larger, we
need to split the computation into several Layer Norms.  Other oddity
to consider for the space constraints is the addressable space, often
we need to read at least 4 Bytes (if not 32B) so in L1 we need to read
at least two columns (reducing $M$).



\subsection{Tiling: $gamma$ and $beta$ }
Notice in Algorithm \ref{alg:l2_4} and \ref{alg:l2_6}, the input is
$\Vc{X}_{i,j}$, it is read twice, it is written once and normalized
$\Vc{X}_{i,j}*\gamma_j + \beta_j$. For a broad cast communication
$\Vc{X}_{i,j}$ takes at least space $4\times k$, we need $2\times k$
for the parameters $\gamma$ and $\beta$, and the partial results for
the reduction (for these norms) is about $2$. We can see that the
tiling of the $\Vc{X}$ really drives the tiling of the the weights. In
fact, consider
\begin{equation}
\dot\sum_k^x\sum_{i}^y\sum_{m}^z\Vc{X}^{k}_{i,m}.
\end{equation}
then the column partition will be the same
\begin{equation}
  \dot\sum_k^x\sum_{i}^y\sum_{m}^z\Vc{{\gamma|\beta}}^{k}_{i,m}.
\end{equation}




\subsection{Tiling: Computation Validation}
For simplicity, assume we have a Tiling and remember that the
temporary space for the partial projection computation will be
actually done in core without spills.
\begin{itemize}
\item $\dot\sum_k^r\sum_{i}^r\sum_{m}^r\Vc{X}^{k}_{i,m}$. 
\begin{verbatim}
# ----
for k in K: 
    for i in I do: 
        for m in J: 
            T= Projection(X[k,i,m]) 
            T = S(T)
            X[k,i,m] = X[k,i,m]/T
# ----
\end{verbatim} 
\item $\dot\sum_k^r\sum_{i}^r\sum_{m}^c\Vc{X}^{k}_{i,m}$. 
\begin{verbatim}
# ----
for k in K: 
    for i in I do: 
        T = 0 # the first time we start the projection      
        for m in J: 
            T+= Projection(X[k,i,m]) 
        T = S(T) # the first time we start the normalization
        for m in J: 
            X[k,i,m] = X[k,i,m]/T
# ----
\end{verbatim} 
\item $\dot\sum_k^r\sum_{i}^c\sum_{m}^c\Vc{X}^{k}_{i,m}$. 
\begin{verbatim}
# ----
for k in K: 
    T = 0# the first time we start the projection 
    for i in I do: 
        for m in J: 
            T+= Projection(X[k,i,m]) 
    T = S(T)  # the first time we start the normalization
    for i in I do: 
        for m in J: 
            X[k,i,m] = X[k,i,m]/T
# ----
\end{verbatim} 
\item $\dot\sum_k^c\sum_{i}^c\sum_{m}^c\Vc{X}^{k}_{i,m}$. 
\begin{verbatim}
# ----
T=0 # the first time we start the projection
for k in K: 
    for i in I do: 
        for m in J: 
            T+= Projection(X[k,i,m]) 
T = S(T) # the first time we start the normalization
for k in K: 
    for i in I do: 
        for m in J: 
            X[k,i,m] = X[k,i,m]/T
# ----
\end{verbatim}
\end{itemize}

In practice, we formulate the computation as a recursive descent visit
into the tiling data structure. Instance Norm has parallel computation
by columns and other Norms  are by row. 



\section{Unified Implementation}

We consider the practical implementation of the computation
$\dot\sum_k^r\sum_{i}^c\sum_{m}^c\Vc{X}^{k}_{i,m}$ that we call
affectionately a two-pass in L2 computation. 
\begin{verbatim}
# ----
for k in K: 
    T = 0# the first time we start the projection 
    for i in I do: 
        for m in J: 
            T+= Projection(X[k,i,m]) 
    T = S(T)  # the first time we start the normalization
    for i in I do: 
        for m in J: 
            X[k,i,m] = X[k,i,m]/T
# ----
\end{verbatim} 
We have a matrix $\Vc{X}$ and we split in parallel set of rows:
$\dot\sum_k^r\Vc{X}^{k}$ but we do not have the resources to execute
them in parallel and thus we can split the computation in
time:$\sum_k^r\Vc{X}^{k}$ and there will be $k\in K$ computations.

The computation is $K$ times:
\begin{itemize}
\item $\Vc{T} = P_+(\Vc{X}^k)$
\item $\Vc{S} = S_+(\Vc{T})$
\item $\Vc{Y}^k = N_*(\Vc{X}^k,\Vc{S})$
\end{itemize}
Implicitly we need to read $\Vc{X}^k$ twice and we can represent such
a sequence as
{\small
\begin{verbatim}
# ----
# L3 
for k in K:  # L2 it can be more complex  
    for iota in [1,2]: # repetition 2 
        if iota==1 : T = 0# the first time we start the projection 
        # L2 -> L1 traversal 
        for i in I do:  
            for m in J:  # Core computation L1
                if iota==1:          T+= P+(X[k,i,m]) 
                if iota==1 and i==1: T = S(T)  # first time normalization
                if iota==2:          Y[k,i,m] = N*(X[k,i,m],T)
# ----
\end{verbatim} }
Note that the inner loop represent the core computation and the
control is based on counting the number of iterations.

This is the mechanical description of the computation. The tiling
shapes and sizes are actually computed bottom up. We compute the
largest problem fitting the core or L1 data cache, then we use this
information to compute the problem fitting L2.





\section{Conclusion}
In this work, we show that Norms can be abstracted in such a way that
mathematical and computation properties can be easily represented.

\begin{enumerate}
  \item We show that Tiling (for AIE) of every Norm can be generated
    by a single heuristics.
  \item The tiling is correct,
  \item We can perform actual computations to validate in practice the
    solution
  \item the code for this project shows that the analysis help in the
    writing of concise codes.
\end{enumerate}

  

%\input{conclusion.tex}

%%%%%%%%% -- BIB STYLE AND FILE -- %%%%%%%%
\bibliographystyle{IEEETran} \bibliography{ref}
%%%%%%%%%%%%%%%%%%%%%%%%%%%%%%%%%%%%

%\appendix{Review and Response}
%\input{review.tex}
\end{document}




  


  




  


