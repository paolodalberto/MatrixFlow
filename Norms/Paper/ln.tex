%% This is file `sample-manuscript.tex', generated with the docstrip
%% utility.  The original source files were: samples.dtx (with
%% options: `manuscript') IMPORTANT NOTICE: For the copyright see the
%% source file.  Any modified versions of this file must be renamed
%% with new filenames distinct from sample-manuscript.tex.  For
%% distribution of the original source see the terms for copying and
%% modification in the file samples.dtx.  This generated file may be
%% distributed as long as the original source files, as listed above,
%% are part of the same distribution. (The sources need not
%% necessarily be in the same archive or directory.)  The first
%% command in your LaTeX source must be the \documentclass command.
\documentclass[acmsmall]{acmart}
%\usepackage{bm}

%% as manuscript for review this will create line number 
\usepackage{lineno}
\usepackage{algorithm}
\usepackage{algorithmic}
%\linenumbers

%%  Always include hyperref last
%% \usepackage[bookmarks=true,breaklinks=true,letterpaper=true,colorlinks,linkcolor=black,citecolor=blue,urlcolor=black]{hyperref}

% Paolo's definition. Yes, I have my way to define stuff. Please take
% a look
%\documentclass[11pt]{article} 
%\usepackage{times}

%\usepackage{pifont}
%\usepackage{bm}
%\usepackage{epsfig}
%\usepackage{psfig}
%\usepackage{tabularx}
%\usepackage{amssymb}
%\usepackage{latexsym}
%\usepackage{amsmath}
%\usepackage{delarray}
%\usepackage{moreverb}
%\usepackage{xspace}
%\usepackage{algorithm}
%\usepackage{algorithmic}

%\usepackage{lscape}
%\usepackage{changebar}
%\usepackage{graphicx}
%\usepackage{graphics}
%\usepackage{mflogo}
%\usepackage{xspace}
%\usepackage{texnames}
%\usepackage{rotating}
%\usepackage{alltt}

%\usepackage{amsmath}
%\usepackage{epsfig}
%\usepackage{booktabs,paralist}

\def\firmopsa{d}
\def\firmopsb{d}
\newcommand\AssignTags{TagIt}


% Definitions
% -----------
\def\x{{\mathbf x}}
\def\L{{\cal L}}
\def\spiral{SPIRAL\xspace}

\newcommand{\pred}{{}}

\newcommand{\tensor}[0]{\otimes}
\newcommand{\inv}[1]{{1{/}{#1}}}
%\newcommand{\inv}[1]{$\textstyle\frac{1}{#1}$}
\newcommand{\DFT}{\operatorname{\bf DFT}}
\newcommand{\WHT}{\operatorname{\bf WHT}}
\newcommand{\FIR}{\operatorname{\bf FIR}}
\newcommand{\one}[0]{{I}}
\newcommand{\eq}[0]{{=}}
\newcommand{\Tensor}[2]{#1{\tensor}#2}
\newcommand{\TensorT}[3]{{#1{\tensor}#2}_{#3}}
\newcommand{\SPw}[2]{{\prec}#1,#2{\succ}}
\newcommand{\SPr}[2]{{<}#1,#2{>}}
\newcommand{\Smax}[1]{{\cal S}_{max}({#1})}
\newcommand{\Smaxd}[1]{\frac{\exp{#1_{i,j}}}{\sum_j\exp{#1_{i,j}}}}
%\newcommand{\Smaxd2}[1]{\frac{\exp{#1_{0,j}}}{\sum_j\exp{#1_{i,j}}}}

\newcommand{\WTensorI}[2]{\Tensor{\WHT_{#1}}{\one_{#2}}}
\newcommand{\ITensorW}[2]{\Tensor{\one_{#1}}{\WHT_{#2}}}
\newcommand{\Wht}[2]{(\WTensorI{\WHT_{#1}}{#2})(\ITensorW{#1}{#2})}

\newcommand{\Times}[2]{{#1}{\times}{#2}}
\newcommand{\EQ}{{=}}
\newcommand{\Space}{{ S}}
\newcommand{\PEQ}{{{+}{=}}}
\newcommand{\SEQ}{{{-}{=}}}
\newcommand{\h}{\psi}
\newcommand{\balpha}{\bm{\alpha}}
\newcommand{\bgamma}{\bm{\gamma}}
\newcommand{\bbeta}{\bm{\beta}}
\newcommand{\bmu}{\bm{\mu}}
\newcommand{\bsigma}{\bm{\sigma}}


\newcommand{\f}{\varphi}
\newcommand{\Z}{\mathbb{Z}}
\newcommand{\W}{\mathbb{W}}
\newcommand{\A}{\mathbb{A}}
\newcommand{\B}{{\bf B}}
\newcommand{\X}{\mathbb{X}}
\newcommand{\T}{\mathbb{T}}
\newcommand{\QQ}{\mathbb{Q}}
\newcommand{\N}{\mathbb{N}}
\newcommand{\R}{\mathbb{R}}
\newcommand{\C}{\mathbb{C}}
\newcommand{\CC}{{\Gamma}}
\newcommand{\D}{\mathbb{D}}
\newcommand{\M}{\mathbb{M}}
\newcommand{\F}{\mathbb{F}}
\newcommand{\ST}{\mathbb{S}}
\newcommand{\Vc}[1]{{\boldsymbol #1}}
\newcommand{\fl}[1]{{\lfloor {#1} \rfloor}}
\newcommand{\cl}[1]{{\lceil {#1} \rceil}}
\newcommand{\half}[1]{{\frac{#1}{2}}}
\newcommand{\q}[1]{\Times{\cl{#1}}{\cl{#1}}}
\newcommand{\qq}[1]{\Times{\cl{#1}}{\fl{#1}}}
\newcommand{\qqq}[1]{\Times{\fl{#1}}{\cl{#1}}}
\newcommand{\qqqq}[1]{\Times{\fl{#1}}{\fl{#1}}}
\newcommand{\size}[1]{\sigma(#1)}
\newcommand{\Size}[3]{\sigma(\Vc{#1})\EQ\Times{#2}{#3}}
\newcommand{\n}{(\frac{n}{2})}
\newcommand{\m}{(\frac{n-1}{2})}
\newcommand{\Ceil}[1]{{\cl{\half{#1}}}}
\newcommand{\Floor}[1]{{\fl{\half{#1}}}}


\newcommand{\Inner}[2]{{\sum_j\fl{\half{#1}}}}





\newcommand{\Endash}{{--}}
\newcommand{\Emdashbegin}{{---}}
\newcommand{\Emdashend}{{---}~}

\newcommand{\Transducer}[1] {{\mathbf #1 }}
\newcommand{\Triangle}[1] {{\vartriangle\!\! #1 }}


\newcommand{\AO}  {\Times{\Ceil{m}}{\Ceil{n}}}
\newcommand{\AI} {\Times{\Ceil{m}}{\Floor{n}}}
\newcommand{\AII}{\Times{\Floor{m}}{\Ceil{n}}}
\newcommand{\AIII} {\Times{\Floor{m}}{\Floor{n}}}
\newcommand{\BO}  {\Times{\Ceil{n}}{\Ceil{p}}}
\newcommand{\BI} {\Times{\Ceil{n}}{\Floor{p}}}
\newcommand{\BII}{\Times{\Floor{n}}{\Ceil{p}}}
\newcommand{\BIII} {\Times{\Floor{n}}{\Floor{p}}}
\newcommand{\CO}  {\Times{\Ceil{m}}{\Ceil{p}}}
\newcommand{\CI} {\Times{\Ceil{m}}{\Floor{p}}}
\newcommand{\CII}{\Times{\Floor{m}}{\Ceil{p}}}
\newcommand{\CIII} {\Times{\Floor{m}}{\Floor{p}}}

\newcommand{\I}[2]{\Times{\Ceil{#1}}{\Ceil{#2}}}
\newcommand{\II}[2]{\Times{\Ceil{#1}}{\Floor{#2}}}
\newcommand{\III}[2]{\Times{\Floor{#1}}{\Ceil{#2}}}
\newcommand{\IV}[2]{\Times{\Floor{#1}}{\Floor{#2}}}


\newcommand{\Q}[2]{\Vc{#1}_{#2}}

\newcommand{\QAO}{{\Vc{A}_{0}}}
\newcommand{\QAI}{\Vc{A}_{1}}
\newcommand{\QAII}{\Vc{A}_{2}}
\newcommand{\QAIII} {\Vc{A}_{3}}

\newcommand{\QBO}  {\Vc{B}_{0}}
\newcommand{\QBI} {\Vc{B}_{1}}
\newcommand{\QBII}{\Vc{B}_{2}}
\newcommand{\QBIII} {\Vc{B}_{3}}

\newcommand{\QCO}  {\Vc{C}_{0}}
\newcommand{\QCI} {\Vc{C}_{1}}
\newcommand{\QCII}{\Vc{C}_{2}}
\newcommand{\QCIII} {\Vc{C}_{3}}
\newcommand{\FIGsize}[2]{{width={#1}, height={#2}}}

\newcommand{\Mod}[2]{{#1}{\text{ \bf mod }}{#2}}

\newcommand{\TableRef}[1]{Table \ref{#1}}
\newcommand{\FigureRef}[1]{Figure \ref{#1}}
\newcommand{\SectionRef}[1]{Section \ref{#1}}
\newcommand{\EquationRef}[1]{Equation \ref{#1}}
\newcommand{\OnPageRef}[1]{on page \pageref{#1}}

% Formatting
% ----------
%\setlength{\evensidemargin}{0mm}
%\setlength{\oddsidemargin}{0mm}
%\setlength{\textwidth}{6.5in}
%\setlength{\textheight}{9.5in}
%\setlength{\topmargin}{0in}
%\setlength{\headheight}{0in}
%\setlength{\headsep}{0mm}

\newenvironment{noinds_itemize}{\begin{list}{$\bullet$}
{\setlength{\rightmargin}{0em}
\setlength{\leftmargin}{1.2em}
\setlength{\itemsep}{0em}
\setlength{\topsep}{0em}
\setlength{\parsep}{0em}}}{\end{list}}



\newcommand{\mypar}[1]{{\bf #1.}}
\newcommand{\mytag}[2]{[#1]_{#2}}
\sloppy

%% jan 01, 2009 - dasdan - removed 'final' pdflatex cannot work with it.
%\usepackage[final=true,bookmarks=true,bookmarkstype=toc,colorlinks=true, linkcolor=blue,citecolor=blue]{hyperref}
%\usepackage[bookmarks=true,bookmarkstype=toc,colorlinks=true, linkcolor=blue,citecolor=blue]{hyperref}

% width fraction, mflops, relative, caption, label
\newcommand{\doublefigure}[5]{{\begin{figure}%
\centering %
\includegraphics[width=#1\linewidth]{#2}
\includegraphics[width=#1\linewidth]{#3}%\figurebox{#1\linewidth}{}{}[#2]
%\figurebox[#1\linewidth]{}{}[#3] %
\caption{#4}%bio                 
\label{#5}%
\end{figure}}}

\newcommand{\Doublefigure}[5]{{\begin{figure*} %
\centering %
%\includegraphics[height=#1\linewidth,angle=-90]{#2}%
%\includegraphics[height=#1\linewidth,angle=-90]{#3} %
\includegraphics[width=#1\linewidth]{#2} \\  %
\includegraphics[width=#1\linewidth]{#3}%
%\figurebox{#1\linewidth}{}{}[#2]
%\figurebox{#1\linewidth}[#3] %
\caption{#4}%
\label{#5}%
\end{figure*}}}

\newcommand{\DDoublefigure}[5]{{\begin{figure*} %
\centering %
%\includegraphics[height=#1\linewidth,angle=-90]{#2}%
%\includegraphics[height=#1\linewidth,angle=-90]{#3} %
\includegraphics[width=#1\linewidth, angle=90]{#2}   %
\includegraphics[width=#1\linewidth, angle=90]{#3}%
%\figurebox{#1\linewidth}{}{}[#2]
%\figurebox{#1\linewidth}[#3] %
\caption{#4}%
\label{#5}%
\end{figure*}}}

\newcommand{\DDDoublefigure}[5]{{\begin{figure}[htb] %
\centering %
%\includegraphics[height=#1\linewidth,angle=-90]{#2}%
%\includegraphics[height=#1\linewidth,angle=-90]{#3} %
\includegraphics[width=#1\linewidth, angle=90]{#2}   %
\includegraphics[width=#1\linewidth, angle=90]{#3}%
%\figurebox{#1\linewidth}{}{}[#2]
%\figurebox{#1\linewidth}[#3] %
\caption{#4}%
\label{#5}%
\end{figure}}}

\newcommand{\Singlefigure}[4]{{\begin{figure*} %
\centering %
\includegraphics[width=#1\linewidth]{#2}%\figurebox{#1\linewidth}{}{}[#2]
\caption{#3}%
\label{#4}%
\end{figure*}}}

\newcommand{\SSinglefigure}[4]{{\begin{figure} %
\centering %
\includegraphics[width=#1\linewidth,angle=90]{#2}%\figurebox{#1\linewidth}{}{}[#2]
\caption{#3}%
\label{#4}%
\end{figure}}}

\newcommand{\singlefigure}[4]{{\begin{figure}[htb] %
\centering %
\includegraphics[width=#1\linewidth]{#2}%
%\includegraphics[height=#1\linewidth,angle=-90]{#3} %
%\figurebox{#1\linewidth}{}{}[#2]\\
\caption{#3}%
\label{#4}%
\end{figure}}}

\newcommand{\singlefigurerotate}[4]{{\begin{figure}[htb] %
\centering %
\includegraphics[width=#1\linewidth,angle=-90]{#2}%
%\includegraphics[height=#1\linewidth,angle=-90]{#3} %
%\figurebox{#1\linewidth}{}{}[#2]\\
\caption{#3}%
\label{#4}%
\end{figure}}}

\newcommand{\orthogonal}{%
\mathrel{\raisebox{.1em}{%     
\reflectbox{\rotatebox[origin=c]{90}{$\models$}}}}}


%%
%% \BibTeX command to typeset BibTeX logo in the docs
\AtBeginDocument{%
  \providecommand\BibTeX{{%
    \normalfont B\kern-0.5em{\scshape i\kern-0.25em b}\kern-0.8em\TeX}}}

%% Rights management information.  This information is sent to you
%% when you complete the rights form.  These commands have SAMPLE
%% values in them; it is your responsibility as an author to replace
%% the commands and values with those provided to you when you
%% complete the rights form.
\setcopyright{acmcopyright}
\copyrightyear{2023}
\acmYear{2023}
\acmDOI{}

%% These commands are for a JOURNAL article.
\acmJournal{JACM}
\acmVolume{--}
\acmNumber{--}
\acmArticle{---}
\acmMonth{3}


%% These commands are for a PROCEEDINGS abstract or paper.
%\acmConference[Woodstock '18]{Woodstock '18: ACM Symposium on Neural
%  Gaze Detection}{June 03--05, 2018}{Woodstock, NY}
%\acmBooktitle{Woodstock '18: ACM Symposium on Neural Gaze Detection,
%  June 03--05, 2018, Woodstock, NY}
%\acmPrice{15.00}
%\acmISBN{978-1-4503-XXXX-X/18/06}


%%
%% Submission ID.
%% Use this when submitting an article to a sponsored event. You'll
%% receive a unique submission ID from the organizers
%% of the event, and this ID should be used as the parameter to this command.
%%\acmSubmissionID{123-A56-BU3}

%%
%% The majority of ACM publications use numbered citations and
%% references.  The command \citestyle{authoryear} switches to the
%% "author year" style.
%%
%% If you are preparing content for an event
%% sponsored by ACM SIGGRAPH, you must use the "author year" style of
%% citations and references.
%% Uncommenting
%% the next command will enable that style.
%%\citestyle{acmauthoryear}

%%
%% end of the preamble, start of the body of the document source.
\begin{document}

%%%%%%%%%%%---SETME-----%%%%%%%%%%%%%
\title{Norms: A Unified Approach}

\author{Paolo D'Alberto}
\email{}
\affiliation{%
  \institution{AMD}
  \streetaddress{2100 Logic Dr}
  \city{San Jose}
  \state{California}
  \postcode{95124}
}

\renewcommand{\shortauthors}{D'Alberto et al.}

\begin{abstract}

  Layer Norms are important tools in today's AI systems.  Also, they
  are the little pesky layers that inference-engine tools have often
  problems in either fusing with other layers for speed-up or
  introducing of numerical unwanted oddities.  Norms have linear
  complexity, they are communication bound, they require to read the
  input at least twice, and there is a satisfying asymmetry of the
  computation (i.e., row Gaussian normalization and column scaling).
  
  Here, we formalize the norms using a unified formalism. We do so to
  clarify the general computation, the opportunity to fuse other
  computations, to highlight non obvious properties and give an
  unified computation model. So we can generate consistent and correct
  code for our AIE AMD computes.
  
\end{abstract}

\maketitle

\section{Some Mathematic definitions} 
\label{sec:introduction}
Here, we are interested only in matrices $X \in \R^{M \times
  N}$. Together with this paper, there will be code implementing and
compute every expression presented here for matrices.

A matrix norm is composed into two parts. We have a {\em projection}
where we compute a factor, a vector, and then a {\em normalization}
where we take the projected factor and distribute it back. In the
literature and implementations may appears with different names:
Projection is often call reduction because it is frequently associated
with sums, dot products. Normalization is often call broadcast because
it is very often based on a Matrix by scalar operations. Now that we
have a name and association to other well known term, let us define
them formally.


We need an operation that address scalar operations:
\begin{definition}
  A scalar operation
  \begin{equation}
    G_f: X \rightarrow Y \text{ and }  y_{i,j} = f(x_{i,j})
  \end{equation}
\end{definition}
This could be used for both projection and normalization. For example,
we will use $x_{i,j}^2$ and $\exp(x_{i,j})$


\begin{definition}
An associative projection $P_+: \Vc{X}\in\R^{M\times N} \rightarrow
\Vc{F} \in \R^{M}$, is a function where the element $f_i =
\sum_{j=1}^N x_{i,j}$. The $+/\sum$ sign is not in general a
summation, but it is a natural for the associative property we
require: $f_i = (x_{i,1} + (\sum_{j=2}^N x_{i,j})) =
((\sum_{j=1}^{N-1} x_{i,j}) + x_{i,N}) $.
\end{definition}

Although the definition is assuming a row projection, we present at
least one projection by columns.  Let us digest our formalism a
little.  As we will read our matrix, we are going to maintain the
order and projecting of matrix $\Vc{X}$ (row elements $x_{i,*}$). We
do not need the projection to be commutative in general but all the
projections of interests are. Also, we require the computation of
multiple projections for the computation of the normalization factor.

We abuse the notation of the symbol $+$ in the associative expression
$f_i = (x_{i,1} + (\sum_{j=2}^N x_{i,j}))$, in the sense that the
combination of the temporary sub-projection may require some extra
work during the computation or at the end when we have a complete
projection into a proper normalization factor.

We present, $P_+(G_f(X))$. That is, we apply scalar operations and
then projection. The associativity is a property of the projection and
not of the value of the matrix.


\begin{definition}
  An optional summary function $S_+: \{\Vc{F}\}_i\in\R^{ M}
  \rightarrow \Vc{E} \in \R^{M}$
\end{definition}
So we can summarize the first part of the computation by an
associative projection (or a set of projections) and an optional
summary function for the final composition of the normalization
factors.


\begin{definition}
  We have two normalization functions by row and by column
  \begin{equation}
    N_*: \Vc{X}\in\R^{M\times N}, \Vc{T}\in \R^M \rightarrow \Vc{Y}
    \in \R^{M\times N}, \text{ where } y_{i,j} = x_{i,j}*t_i
  \end{equation}
  \begin{equation}
    N_*: \Vc{X}\in\R^{M\times N}, \Vc{S}\in \R^N \rightarrow \Vc{Y}
    \in \R^{M\times N}, \text{ where } y_{i,j} = x_{i,j}*s_j
  \end{equation}
  Again the $*$ is not necessarily a multiplication and the
  normalization can be by row or by column. For presentation purpose
  we will use a single symbol unless it will be confusing.
\end{definition}

If we extend the normalizing factor $\Vc{F}$ into a diagonal matrix in
$\Vc{D}_F \in  \R^{M\times M}$ (or $\R^{N\times N}$), then the normalization above
are the following matrix products

\begin{equation}
  \Vc{D}_f * \Vc{X} \text{ and }  \Vc{X} * \Vc{D}_f 
\end{equation}
Notice that using matrix multiplication we can see that a row
normalization is associative $(\Vc{D}_f * \Vc{X}) * \Vc{Y} = \Vc{D}_f
* (\Vc{X} * \Vc{Y})$ then giving an hint how the normalization could
be postponed if neceesary.  Also a column mormalization in combination
with a matrix multiplication $(\Vc{X}* \Vc{D}_f) * \Vc{Y} = \Vc{X}*
(\Vc{D}_f * \Vc{Y})$ is actually a row normalization of the following
operand.

We introduce the last definiition, a matrix partition.
\begin{definition}
  A row partition of a matrix is simply defined as
  \begin{equation}
    [\Vc{X} ]_{r=2} = \
    \begin{pmatrix}
      \Vc{X}_1  \\
      \Vc{X}_2  \\
    \end{pmatrix}\\
  \end{equation}
  and a column partition
  \begin{equation}
    [\Vc{X} ]_{c=2} = \
    \begin{pmatrix}
      \Vc{X}_1  & \Vc{X}_2  \\
    \end{pmatrix}\\
  \end{equation}
\end{definition}


If we want to take a matrix $\Vc{X} \in \R^{M\times N}$ and normalize
its row by a square norm:
\begin{algorithm}
    \caption{$\Vc{X}/||\Vc{X}||_2$ algorithm}
    \label{alg:l2}
    \begin{algorithmic}
      \STATE $\Vc{T} = P_+(G_{x*x}(\Vc{X}))$\\
      \STATE $\Vc{T} = \sqrt{\Vc{T}/N}$    \\
      \STATE $\Vc{X} = N_*(\Vc{X},\Vc{T})$ \\
    \end{algorithmic}
\end{algorithm}
The above algorithm is just a rework of the definition providing
little extra information.  As soon as we define $P_+(\Vc{X})_i =
\sum_j x_{i,j}$ and $N_*(\Vc{X}, \Vc{T})_{i,j} = x_{i,j}*t_i$, we have
the complete computation of the norm.

\begin{algorithm}
    \caption{$SoftMax(\Vc{X})$ algorithm}
    \label{alg:l2_2}
    \begin{algorithmic}
      \STATE $\Vc{M} = P_{max}(\Vc{X})$\\
      \STATE $\Vc{T} = P_+(G_{exp}(\Vc{X})*G_{exp}(-\Vc{M}))$\\
      \STATE $\Vc{T} = \Vc{T}*\exp{\Vc{M}}$    \\
      \STATE $\Vc{X} = N_*(G_{exp}(\Vc{X}),\Vc{T})$ \\
    \end{algorithmic}
\end{algorithm}

When we introduce a new projections $P_{max}(\Vc{X})_i =
\max{x_{i,*}}$, which is associative, and we normalize the additive
projection with the result of the new projection, we can see that the
Softmax computation has the same properties and composition.

Let us go back using the original notations and block the input matrix
so that to exploit its natural composition.



\begin{algorithm}
    \caption{Blocked $3\times 3$}
    \label{alg:l2_3}
    \begin{algorithmic}
      \STATE \[
        \Vc{X}  =   \begin{pmatrix}
          \Vc{X}_{1,1} &    \Vc{X}_{1,2}&    \Vc{X}_{1,3}  \\
          \Vc{X}_{2,1} &    \Vc{X}_{2,2}&    \Vc{X}_{2,3}  \\
          \Vc{X}_{3,1} &    \Vc{X}_{3,2}&    \Vc{X}_{3,3}  \\
        \end{pmatrix}
      \] \\
      \WHILE{$i$ in [1,2,3] }
          \STATE $\Vc{T} = \Vc{0} $  \COMMENT{Zero element for $P_+$}
          \WHILE {$j$ in [1, 2,3] }
              \STATE $\Vc{T} = \Vc{T} +P_+(G_{f}(\Vc{X}_{i,j}))$ \COMMENT{Projection + Reduction}  
          \ENDWHILE
          \STATE $\Vc{T}  =   S_+(\Vc{T})$       \COMMENT{Summary, scalar}
          \WHILE {$j$ in [1,2,3] }
             \STATE $ \Vc{X}_{i,j} = N_*(\Vc{X}_{i,j},\Vc{T})$  \COMMENT{Normalization}
          \ENDWHILE
      \ENDWHILE
    \end{algorithmic}
\end{algorithm}

The application of the blocked algorithm for the Euclidean Norm and
for Softmax requires very little re-work, just remember that softmax
will require two projections:

\begin{equation} \Vc{M}_0 = -\infty; \Vc{T}_0 = 0 \end{equation}
\begin{equation} \Vc{M}_{j+1} = max(M_j, P_{max}(\Vc{X}_{i,j}) \end{equation}
\begin{equation} \Vc{T}_{j+1} = \Vc{T}_j*(exp(-(\Vc{M}_{j+1}-\Vc{M}_j))) + P_+(G_{\exp}(\Vc{X}_{i,j}-\Vc{M}_{j+1})) \end{equation}

But once the obstacle of recognizing the particular projection is
surmounted, the matching of the proper computation should be
natural. Also, the layer norm $(x_{i,j} - \mu_i)/\sqrt{\sigma^2_i}$
(without column scaling) will follow the softmax algorithm because it
will require the computation of two projections so that average
($\mu$) and variance ($\sigma^2$) can be computed in a single pass (Algorithm \ref{alg:l2_4}).

\begin{algorithm}
    \caption{Layer Norm $3\times 3$ with $\gamma, \beta$}
    \label{alg:l2_4}
    \begin{algorithmic}
      \STATE \[
        \Vc{X}  =   \begin{pmatrix}
          \Vc{X}_{1,1} &    \Vc{X}_{1,2}&    \Vc{X}_{1,3}  \\
          \Vc{X}_{2,1} &    \Vc{X}_{2,2}&    \Vc{X}_{2,3}  \\
          \Vc{X}_{3,1} &    \Vc{X}_{3,2}&    \Vc{X}_{3,3}  \\
        \end{pmatrix}
      \] 
      \STATE \[
        \Vc{\gamma}  =   \begin{pmatrix}
          \Vc{\gamma}_{1} &    \Vc{\gamma}_{2}&    \Vc{\gamma}_{3}  \\
        \end{pmatrix}
      \] 
      \STATE \[
        \Vc{\beta}  =   \begin{pmatrix}
          \Vc{\beta}_{1} &    \Vc{\beta}_{2}&    \Vc{\beta}_{3}  \\
        \end{pmatrix}
      \] 
      \WHILE{$i$ in [1,2,3] }
          \STATE $\Vc{T} = \Vc{0} $ 
          \STATE $\Vc{S} = \Vc{0} $ 
          \WHILE {$j$ in [1, 2,3] }
              \STATE $\Vc{T} = \Vc{T} +P_+(\Vc{X}_{i,j})$             \COMMENT{Projection + sum}  
              \STATE $\Vc{S} = \Vc{T} +P_+(G_{x*x}(\Vc{X}_{i,j}))$ \COMMENT{Projection + sum squares}  
          \ENDWHILE
          \STATE $\Vc{T}  =   \Vc{T}/N$       \COMMENT{Summary, scalar}
          \STATE $\Vc{S}  =   (\Vc{S} - \Vc{T}^2/N)/N$       \COMMENT{Summary, scalar}
          \WHILE {$j$ in [1,2,3] }
             \STATE $ \Vc{X}_{i,j} = \Vc{X}_{i,j}-\Vc{T}$  \COMMENT{Normalization Row}
             \STATE $ \Vc{X}_{i,j} = \Vc{X}_{i,j}/\Vc{S}$   
             \STATE $ \Vc{X}_{i,j} = \Vc{X}_{i,j}*\gamma_j$   \COMMENT{Normalization Column}
             \STATE $ \Vc{X}_{i,j} = \Vc{X}_{i,j} + \beta_j$   
          \ENDWHILE
      \ENDWHILE
    \end{algorithmic}
\end{algorithm}


\section{Computation Modes} 
The mathematical notation provides a clear description of the
computation and it two pass nature. Also, the normalization factor and
its computation can be used when we would like to fuse the norm
operation with the following (or the previous) computation.

Here, we are more interested in describing how we would implement the
norm computation using different engine: GPU, GPU, and AIE FGPA
engines. We show direct implementations for CPU and AIE and we present
a short discussion about the GPUs. This is always related to AMD chips
and codes.

The main difference is about the computation of the Projection and how
the hardware implicitly helps the computation. Let's start with the
GPU that uses heavily the associativity for parallelization of the
computation.

\subsection{GPU conversation}
The projection is associative and we can exploit the parallelism by
blocking the computation in warps, compute independent and separated
$\Vc{T}_i = P_+(\Vc{X}_{i,j})$, then reduce the computation in cores
by using a tree like computation and built in HW utilities.

\begin{equation}
  \begin{matrix}
    \Vc{T}_1 = P_+(\Vc{X}_{i,1}), & ... &  \Vc{T}_n = P_+(\Vc{X}_{i,n}) \\ 
    \Vc{T}_1 = \Vc{T}_1+\Vc{T}_2, & ... &  \Vc{T}_{\frac{n}{2}} = \Vc{T}_{\frac{n}{2}} + \Vc{T}_{\frac{n}{2}+1} \\
    \Vc{T}_1 = \Vc{T}_1 + \Vc{T}_2, & ... &  \Vc{T}_{\frac{n}{4}} = \Vc{T}_{\frac{n}{4}} + \Vc{T}_{\frac{n}{4}+1} \\
    \dots  & ... & ... \\ 
    \Vc{T}_1 = \Vc{T}_1 + \Vc{T}_2 & & \\
  \end{matrix}
\end{equation}

The small size of the computation that can be done as unit and the HW
support for the reduction makes this computation
appealing. Numerically, the binary three computation make the
computation balanced and well partitioned. If there is no use of high
precision accumulators this will have numerical advantages and will
make it highly parallel.

For a Layer Norm of size $512 \times 768$ on a MI100 with 1.2 TB/s
HBM2 we can achieve 9 us latency (800GBs) for layer norm in fp16 and
fp32 precision. This is the fastest performance across the AMD
products at our disposal.

\subversion{CPU conversation}
CPUs are highly general and also they have tricks up theirs sleeves
for the efficient computation of inner products by highly vectorizable
extensions such as 

 












\section{Conclusion}


  







%\input{conclusion.tex}

%%%%%%%%% -- BIB STYLE AND FILE -- %%%%%%%%
\bibliographystyle{IEEETran} \bibliography{ref}
%%%%%%%%%%%%%%%%%%%%%%%%%%%%%%%%%%%%

%\appendix{Review and Response}
%\input{review.tex}
\end{document}




  


  




  


